\section{Introduction and Motivation}
In this chapter I will introduce an extension of the celebrated Falikov-Kimball (FK) model. The FK model is a tight binding hamiltonian that models the interaction between two different species of particle. It makes the key approximation that one of the species has no quantum character at all, which makes it more tractable than a generic interacting many-body hamiltonian.

Though the FK model is possibly one of the simplest lattice models you could think up, there is much that is still unknown about it. I will review what had already been learned about the model and identify gaps in the literature.

I will then introduce an extension of the model that allows us to bring phenomena observed in the 2 and 3D FK model down to one dimension where the interplay of dimensionality, localisation and disorder can be explored.

Next I will discuss how the Markov Chain Monte Carlo method introduced in \autoref{sec:mcmc} can be applied to our Long-Range FK model.

Finally I will present my results on the Long-Range FK model.
		
\section{Literature Review}

\subsection{The Falikov-Kimball model}

Historically, a lot of progress was made in condensed matter with the band theory of solids. This theory models the interaction between a single electron and the ionic lattice but it neglects interactions between electrons. From band theory we get the concept of a valence band, a set of occupied electronic states, and a conduction band just above it, containing unoccupied states. Systems are predicted to be insulators when there is a large energy gap between the valence and conduction band and to be conductors when the gap is less that the thermal energy scale \(k_bT\). In many systems band theory is enough to make good predictions about electrical, mechanical and thermal properties~\cite{neilw.ashcroftSolidStatePhysics1976}. 

Later, the development of Landau Fermi Liquid theory explained why band theory works so well even in cases where an analysis of the relevant energies suggests that it should not~\cite{wenQuantumFieldTheory2007}. Landau Fermi Liquid theory demonstrates that in many cases where electron-electron interactions are significant, the system can still be described in terms on generalised non-interacting quasiparticles.

However there are systems where even Landau Fermi Liquid theory fails. An effective theoretical description of these systems must include electron-electron correlations and they are thus called Strongly Correlated Materials~\cite{morosanStronglyCorrelatedMaterials2012}, Correlated Electron systems or Quantum Materials. The canonical class of strongly correlated materials are the transition metal oxides. 

The transition metal oxides have half filled valence bands and hence are predicted to be electrically conductive by band theory. Experimentally they are of course found to be insulators, leading Mott and others in 1937 to suggest that their insulating character is caused by coulomb repulsion between electrons opening an energy gap that prevents conduction~\cite{mottDiscussionPaperBoer1937}.

However a firmer theoretical description of the Mott-transition had to wait until 1963 when Martin Gutzwiller~\cite{gutzwillerEffectCorrelationFerromagnetism1963}, Junjiro Kanamori~\cite{kanamoriElectronCorrelationFerromagnetism1963} and John Hubbard~\cite{hubbardj.ElectronCorrelationsNarrow1963} independently proposed what would become known as the Hubbard Model.

The Hubbard Model is about the simplest interacting tight-binding hamiltonian that could be written down. It is a tight binding model of spin half electrons with finite bandwidth \(t\) and a repulsive on-site interaction \(U > 0\).
\[
    H = -t\sum_{<ij>}c^\dag_{i\sigma}c_{j\sigma} + U \sum_{i} (n_{i \uparrow} - 1/2)( n_{i\downarrow} - 1/2) - \mu \sum_i \left( n_{i \uparrow} + n_{i \downarrow} \right).
\]
Where as usual \(n_{i \sigma} = c^\dag_{i\sigma}c_{i\sigma}\) is the number operator. I'm using the particle-hole symmetric version of the interaction term which would otherwise be written as \(n_{i \uparrow} n_{i\downarrow}\). The difference just amounts to a redefinition of the chemical potential. 

While it was originally used to explain the Mott metal-insulator transition, the Hubbard Model has seen applications to high-temperature superconductivity and become the target for cold-atom optical trap experiments~\cite{noauthor_hubbard_2013, greiner_quantum_2002, jordens_mott_2008}. Only a few analytic results exist, namely the Bethe ansatz \cite{lieb_absence_1968} which proves the absence of even a zero temperature phase transition in the 1D model and Nagaoka’s theorem \cite{nagaoka_ferromagnetism_1966} which proves that the three dimensional model has a ferromagnetic ground state in the vicinity of half filling.

A full theoretical treatment of the Hubbard Model remains elusive because it is an interacting quantum many-body system. Together these three properties make for a formidable theoretical challenge. It is for this reason that many approaches within condensed matter theory today can be interpreted as relaxing one of these three properties. While this may limit the applicability of these new models, they can still yield useful insights.

Let's start with interactions, if we remove them entirely we end up back with band theory. If we make the interactions small we're now able to use perturbative methods to take the interactions into account. This is what is done in 

If we instead relax the requirement that the system be many-body we could say try direct simulations of small numbers of particles. 

Finally we can think about relaxing the requirement that the system be quantum. In the extreme case we could make the system entirely classical and be left with something like the Ising Model. However, there's another option that still yields something more tractable than the Hubbard Model. The trick is that interactions are only problematic if they are between quantum particles. Interactions between a quantum particle and a classical one are much easier to deal with. Now note that in the Hubbard model, electrons with the same spin are prevented from occupying the same site by the Pauli Exclusion Principle so the interactions are only ever between electrons of opposite spin. Now if we imagine splitting the spin half electron into two species and making one of them classsical we arrive directly at the Falikov-Kimball Model: 




The Falikov-Kimball (FK) model is one of the simplest models of the correlated electron problem. It captures the essence of the interaction between itinerant and localized electrons. More plainly, between electons that can move and those that can't.








            Particle hole symmetry
            
\subsubsection{Thermodynamics}
    Phase diagram in 2D 
    
\subsection{The Long-ranged Ising model}
		Renormalisation group phase diagram
		Peierls argument extended to long range terms

\section{The Long-Range Falikov-Kimball Model}
        Model
		Looking at the FK model in 1D
		The long ranged coupling
		Non-extensive hamiltonians

\section{Methods}
		MCMC for models with separated classical and quantum energy terms
		Finite Size Scaling 
        Smoothing the long range term on periodic systems
        Binder cumulants as a probe for phase transitions
		
\section{Results}
		Phase diagram
		Localisation Properties
