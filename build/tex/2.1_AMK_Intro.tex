\begin{Shaded}
\begin{Highlighting}[]
\OperatorTok{\%\%}\NormalTok{html}
\end{Highlighting}
\end{Shaded}

\hypertarget{introduction}{%
\section{Introduction}\label{introduction}}

The Kitaev-Honeycomb model is remarkable because it was the first such
model that combined three key properties.

First, it is a plausible tight binding Hamiltonian. The form of the
Hamiltonian could be realised by a real material. Indeed candidate
materials such as \(\alpha\mathrm{-RuCl}_3\) were quickly found
\autocite{banerjeeProximateKitaevQuantum2016,trebstKitaevMaterials2022}
that are expected to behave according to the Kitaev with small
corrections.

Second, the Kitaev Honeycomb model is deeply interesting to modern
condensed matter theory. Its ground state is almost the canonical
example of the long sought after quantum spin liquid state. Its
excitations are anyons, particles that can only exist in two dimensions
that break the normal fermion/boson dichotomy. Anyons have been the
subject of much attention because, among other reasons, there are
proposals to braid them through space and time to achieve noise tolerant
quantum computations
\textcite{freedmanTopologicalQuantumComputation2003}.

Third and perhaps most importantly, it a rare many body interacting
quantum system that can be treated analytically. It is exactly
solveable. We can explicitly write down its many body ground states in
terms of single particle states
\textcite{kitaevAnyonsExactlySolved2006}. Its solubility comes about
because the model has extensively many conserved degrees of freedom that
mediate the interactions between quantum degrees of freedom.

\textbf{Insert discussion of why a generalisation to the amorphous case
is intersting}

\hypertarget{chapter-outline}{%
\subsection{Chapter outline}\label{chapter-outline}}

In this chapter I will discuss the physics of the Kitaev Model on
amorphous lattices.

I'll start by discussing the physics of the Kitaev model in much more
detail. Here I will look at the gauge symmetries of the model as well as
its solution via a transformation to a Majorana hamiltonian. From this
discusssion we will see that for the the model to be sovleable it need
only be defined on a trivalent, tri-edge-colourable lattice
\autocite{Nussinov2009}.

In the methods section, I will discuss how to generate such lattices and
colour them as well as how to map back and forth between configurations
of the gauge field and configurations of the gauge invariant quantities.

In results section, I will begin by looking at the zero temperature
physics. I'll present numerical evidence that the ground state of the
model is given by a simple rule. I'll make an assessment of the gapless,
abelian and non-abelian phases that are present as well as spontaneous
chiral symmetry breaking and topological edge states. We will also
compare the zero temperature phase diagram to that of the Kitaev
Honeycomb Model. Next I will take the model to finite temperature and
demonstrate that there is a phase transition to a thermal metal state.

In the Discussion I will consider possible physical realisations of this
model as well the motivations for doing so. I will alao discuss how a
well known quantum error correcting code defined on the Kitaev Honeycomb
could be generalised to the amorphous case.

Various generalisations have been made, one mode replaces pairs of
hexagons with heptagons and pentagons
\cite{periNonAbelianChiralSpin2020} and another that replaces vertices
of the hexagons with triangles \cite{yaoExactChiralSpin2007}. When we
generalise this to the amorphous case, the key property that will remain
is that each vertex interacts with exactly three others via an x, y and
z edge. However the lattice will no longer be bipartite, breaking chiral
symmetry among other things.

\hypertarget{kitaev-heisenberg-model}{%
\subsection{Kitaev-Heisenberg Model}\label{kitaev-heisenberg-model}}

In real materials there will generally be an addtional small Heisenberg
term
\[H_{KH} =  - \sum_{\langle j,k\rangle_\alpha} J^{\alpha}\sigma_j^{\alpha}\sigma_k^{\alpha} + \sigma_j\sigma_k\]

\hypertarget{an-in-depth-look-at-the-kitaev-model}{%
\section{An in-depth look at the Kitaev
Model}\label{an-in-depth-look-at-the-kitaev-model}}

\hypertarget{commutation-relations}{%
\subsection{Commutation relations}\label{commutation-relations}}

Before diving into the Hamiltonian of the Kitaev Model, here is a quick
refresher of the key commutation relations of spins, fermions and
Majoranas.

\hypertarget{spins}{%
\subsubsection{Spins}\label{spins}}

Skip this is you're super familiar with the algebra of the Pauli
martrices. Scalars like \(\delta_{ij}\) should be understood to be
multiplied by an implicit identity \(\mathbb{1}\) where necessary.

We can represent a single spin\(-1/2\) particle using the Pauli matrices
\((\sigma^x, \sigma^y, \sigma^z) = \vec{\sigma}\), these matrices all
square to the identity \(\sigma^\alpha \sigma^\alpha = \mathbb{1}\) and
obey nice commutation and exchange rules:
\[\sigma^\alpha \sigma^\beta = \delta^{\alpha \beta} + i \epsilon^{\alpha \beta \gamma} \sigma^\gamma\]
\[[\sigma^\alpha, \sigma^\beta] = 2 i \epsilon^{\alpha \beta \gamma} \sigma^\gamma\]

Adding a sites indices \(ijk...\), spins at different spatial sites
commute always \([\vec{\sigma}_i, \vec{\sigma}_j] = 0\) so when
\(i \neq j\)
\[\sigma_i^\alpha \sigma_j^\beta = \sigma_j^\alpha \sigma_i^\beta\]
\[[\sigma_i^\alpha, \sigma_j^\beta] = 0\] while the previous equations
hold for \(i = j\).

Two extra relations that will be useful for the Kitaev model are the
value of \(\sigma^\alpha \sigma^\beta \sigma^\gamma\) and
\([\sigma^\alpha \sigma^\beta, \sigma^\gamma]\) when
\(\alpha \neq \beta \neq \gamma\) these can be computed quite easily by
appling the above relations yielding:
\[\sigma^\alpha \sigma^\beta \sigma^\gamma = i \epsilon^{\alpha\beta\gamma}\]
and \[[\sigma^\alpha \sigma^\beta, \sigma^\gamma] = 0\]

\hypertarget{fermions-and-majoranas}{%
\subsubsection{Fermions and Majoranas}\label{fermions-and-majoranas}}

The fermionic creation and anhilation operators are defined by the
canonical anticommutation relations \[\begin{aligned}
\{f_i, f_j\} &= \{f^\dagger_i, f^\dagger_j\} = 0\\
\{f_i, f^\dagger_j\} &= \delta_{ij}
\end{aligned}\] which give us the exchange statistics and Pauli
exclusion principle.

From fermionic operators, we can construct Majorana operators:
\[\begin{aligned}
f_i         &= 1/2 (a_i + ib_i)\\
f^\dagger_i &= 1/2(a_i - ib_i)\\
a_i         &= f_i + f^\dagger_i = 2\mathbb{R}f\\
b_i         &= 1/i(f_i - f^\dagger_i) = 2\mathbb{I} f 
\end{aligned}\]

Majorana operators are the real and imaginary parts of the fermionic
operators, physically they correspond to the orthogonal superpositions
of the presence and absence of the fermion and are thus a kind of
quasiparticle.

Once we involve multiple fermions there is quite a bit of freedom in how
we can perform the transformation from \(n\) fermions \(f_i\) to \(2n\)
Majoranas \(c_i\). The property that must be preserved however is that
the Majoranas still anticommute:

\[ \{c_i, c_j\} = 2\delta_{ij}\]

\hypertarget{the-hamiltonian}{%
\subsection{The Hamiltonian}\label{the-hamiltonian}}

To get down to brass tacks, the Kitaev Honeycomb model is a model of
interacting spin\(-1/2\)s on the vertices of a honeycomb lattice. Each
bond in the lattice is assigned a label \(\alpha \in \{ x, y, z\}\) and
that bond couples its two spin neighbours along the \(\alpha\) axis. See
\cref{fig:visual_kitaev_1} for a diagram.

This gives us the Hamiltonian
\[H =  - \sum_{\langle j,k\rangle_\alpha} J^{\alpha}\sigma_j^{\alpha}\sigma_k^{\alpha},\]
where \(\sigma^\alpha_j\) is a Pauli matrix acting on site \(j\) and
\(\langle j,k\rangle_\alpha\) is a pair of nearest-neighbour indices
connected by an \(\alpha\)-bond with exchange coupling \(J^\alpha\)
\textcite{kitaevAnyonsExactlySolved2006}. For notational brevity is is
useful to introduce the bond operators
\(K_{ij} = \sigma_j^{\alpha}\sigma_k^{\alpha}\) where \(\alpha\) is a
function of \(i,j\) that picks the correct bond type.

\begin{figure}
\hypertarget{fig:visual_kitaev_1}{%
\centering
\includegraphics[width=1\textwidth,height=\textheight]{figure_code/amk_chapter/visual_kitaev_1.pdf}
\caption{}\label{fig:visual_kitaev_1}
}
\end{figure}

\ref{fig:vis_kitaev_1}

This Kitaev model has a set of conserved quantities that, in the spin
language, take the form of Wilson loop operators \(W_p\) winding around
a closed path on the lattice. The direction doesn't matter, but I will
stick to clockwise here. I'll use the term plaquette and the symbol
\(\phi\) to refer to a Wilson loop operator that does not enclose any
other sites, such as a single hexagon in a honeycomb lattice.

\[W_p = \prod_{\mathrm{i,j}\; \in\; p} K_{ij} = \sigma_1^z \sigma_2^x \sigma_2^y \sigma_3^y .. \sigma_n^y \sigma_n^y \sigma_1^z\]

\textbf{add a diagram of a single plaquette with labelled site and bond
types}

In closed loops, each site appears twice in the product with two of the
three bond types. Applying
\(\sigma^\alpha \sigma^\beta = \epsilon^{\alpha \beta \gamma} \sigma^\gamma, \alpha \neq \beta\)
then gives us a product containing a single pauli matrix associated with
each site in the loop with the type of the \emph{outward} pointing bond.
From this we see that the \(W_p\) associated with hexagons or shapes
with an even number of sides all square to 1 and hence have eigenvalues
\(\pm 1\).

A consequence of the fact that the honeycomb lattice is bipartite is
that there are no closed loops that contain an even number of
edges\footnote{A bipartite lattice is composed of A and B sublattices
  with no intra-sublattice edges i.e no A-A or B-B edges. Any closed
  loop must begin and at the same site, let's say it's an A site. The
  loop must go A-B-A-B\ldots{} until it returns to the original site and
  must therefore must contain an even number of edges in order to end on
  the same sublattice that it started on.} and hence all the \(W_p\)
have eigenvalues \(\pm 1\) on bipartite lattices. Later we will show
that plaquettes with an odd number of sides (odd plaquettes for short)
will have eigenvalues \(\pm i\).

\begin{figure}
\hypertarget{fig:regular_plaquettes}{%
\centering
\includegraphics[width=0.86\textwidth,height=\textheight]{figure_code/amk_chapter/regular_plaquettes/regular_plaquettes.pdf}
\caption{The eigenvalues of a loop or plaquette operators depend on how
many bonds in its enclosing path.}\label{fig:regular_plaquettes}
}
\end{figure}

Remarkably, all of the spin bond operators \(K_{ij}\) commute with all
the Wilson loop operators \(W_p\). \[[W_p, J_{ij}] = 0\] We can prove
this by considering the three cases: 1. neither \(i\) nor \(j\) is part
of the loop 2. one of \(i\) or \(j\) are part of the loop 3. both are
part of the loop

The first case is trivial while the other two require a bit of algebra,
outlined in \cref{fig:visual_kitaev_2}.

\begin{figure}
\hypertarget{fig:visual_kitaev_2}{%
\centering
\includegraphics[width=1.43\textwidth,height=\textheight]{figure_code/amk_chapter/visual_kitaev_2.pdf}
\caption{}\label{fig:visual_kitaev_2}
}
\end{figure}

Since the Hamiltonian is just a linear combination of bond operators, it
also commutes with the plaquette operators! This is great because it
means that the there's a simultaneous eigenbasis for the Hamiltonian and
the plaquette operators. We can thus work in a basis in which the
eigenvalues of the plaquette operators take on a definite value and for
all intents and purposes act like classical degrees of freedom. These
are the extensively many conserved quantities that make the model
tractable.

Plaquette operators measure flux. We will find that the ground state of
the model corresponds to some particular choice of flux through each
plaquette. I will refer to excitations which flip the expectation value
of a plaqutte operator away from the ground state as \textbf{vortices}.

Fixing a configuration of the vortices thus partitions the many-body
Hilbert space into a set of `vortex sectors' labelled by that particular
flux configuration \(\phi_i = \pm 1,\pm i\).

\hypertarget{from-spins-to-majorana-operators}{%
\subsection{From Spins to Majorana
operators}\label{from-spins-to-majorana-operators}}

\hypertarget{for-a-single-spin}{%
\subsubsection{For a single spin}\label{for-a-single-spin}}

Let's start by considering just one site and its \(\sigma^x, \sigma^y\)
and \(\sigma^z\) operators which live in a two dimensional Hilbert space
\(\mathcal{L}\).

We will introduce two fermionic modes \(f\) and \(g\) that satisy the
canonical anticommutation relations along with their number operators
\(n_f = f^\dagger f, n_g = g^\dagger g\) and the total fermionic parity
operator \(F_p = (2n_f - 1)(2n_g - 1)\) which we can use to divide their
Fock space up into even and odd parity subspaces which are separated by
the addition or removal of one fermion.

From these two fermionic modes we can build four Majorana operators:
\[\begin{aligned}
b^x &= f + f^\dagger\\
b^y &= -i(f - f^\dagger)\\
b^z &= g + g^\dagger\\
c   &= -i(g - g^\dagger)
\end{aligned}\]

The Majoranas obey the usual commutation relations, squaring to one and
anticommuting with eachother. The fermions and Majorana live in a 4
dimenional Fock space \(\mathcal{\tilde{L}}\). We can therefore identify
the two dimensional space \(\mathcal{M}\) with one of the partity
subspaces of \(\mathcal{\tilde{L}}\) which we will call the
\emph{physical subspace} \(\mathcal{\tilde{L}}_p\). Kitaev defines the
operator \[D = b^xb^yb^zc\] which can be expanded out to
\[D = -(2n_f - 1)(2n_g - 1) = -F_p\] and labels the physical subspace as
the space sanned by states for which \[ D|\phi\rangle = |\phi\rangle\]

We can also think of the physical subspace as whatever is left after
applying the projector \[P  = \frac{1 - D}{2}\] to it. This formulation
will be useful for taking states that span the extended space
\(\mathcal{\tilde{M}}\) and projecting them into the physical subspace.

So now, with the caveat that we are working in the physical subspace, we
can define new pauli operators:

\[\tilde{\sigma}^x = i b^x c,\; \tilde{\sigma}^y = i b^y c,\; \tilde{\sigma}^y = i b^y c\]

These extended space pauli operators satisfy all the usual commutation
relations, the only difference being that if we evaluate
\(\sigma^x \sigma^y \sigma^z = i\) we instead get
\[ \tilde{\sigma}^x\tilde{\sigma}^y\tilde{\sigma}^z = iD \]

Which indeed makes sense, as long as we promise to confine ourselves to
the physical subspace \(D = 1\) and this all makes sense.

\begin{figure}
\hypertarget{fig:majorana}{%
\centering
\includegraphics[width=0.71\textwidth,height=\textheight]{figure_code/majorana.png}
\caption{}\label{fig:majorana}
}
\end{figure}

\hypertarget{for-multiple-spins}{%
\subsubsection{For multiple spins}\label{for-multiple-spins}}

This construction generalises easily to the case of multiple spins: we
get a set of 4 Majoranas \(b^x_j,\; b^y_j,\;b^z_j,\; c_j\) and a
\(D_j = b^x_jb^y_jb^z_jc_j\) operator for every spin. For a state to be
physical we require that \(D_j |\psi\rangle = |\psi\rangle\) for all
\(j\).

From these each Pauli operator can be constructed:
\[\tilde{\sigma}^\alpha_j = i b^\alpha_j c_j\]

This is where the magic happens. We can promote the spin hamiltonian
from \(\mathcal{L}\) into the extended space \(\mathcal{\tilde{L}}\),
safe in the knowledge that nothing changes so long as we only actually
work with physical states. The Hamiltonian \[\begin{aligned}
\tilde{H} &=  - \sum_{\langle j,k\rangle_\alpha} J^{\alpha}\tilde{\sigma}_j^{\alpha}\tilde{\sigma}_k^{\alpha}\\
          &= \frac{i}{4} \sum_{\langle j,k\rangle_\alpha} 2J^{\alpha} (ib^\alpha_i b^\alpha_j) c_i c_j\\
          &=  \frac{i}{4} \sum_{\langle i,j\rangle_\alpha} 2J^{\alpha} \hat{u}_{ij} \hat{c}_i \hat{c}_j
\end{aligned}\]

We can factor out the Majorana bond operators
\(\hat{u}_{ij} = i b^\alpha_i b^\alpha_j\). Note that these bond
operators are not equal to the spin bond operators
\(K_{ij} = \sigma^\alpha_i \sigma^\alpha_j = - \hat{u}_{ij} c_i c_j\).
In what follows we will work much more frequently with the Majorana bond
operators so when I refer to bond operators without qualification, I am
refering to the Majorana variety.

Similar to the argument with the spin bond operators \(K_{ij}\) we can
quickly verify by considering three cases that the Majorana bond
operators \(u_{ij}\) all commute with one another. They square to one so
have eigenvalues \(\pm 1\) and they also commute with the \(c_i\)
operators.

Another important point here is that the operators
\(D_i = b^x_i b^y_i b^z_i c_i\) commute with \(K_{ij}\) and therefore
with \(\tilde{H}\). We will show later that the action of \(D_i\) on a
state is to flip the values of the three \(u_{ij}\) bonds that connect
to site \(i\). Physcially this is telling us that \(u_{ij}\) is a gauge
field with a high degree of degeneracy.

In summary Majorana bond operators \(u_{ij}\) are an emergent,
classical, \(\mathbb{Z_2}\) gauge field!

\hypertarget{partitioning-the-hilbert-space-into-bond-sectors}{%
\subsection{Partitioning the Hilbert Space into Bond
sectors}\label{partitioning-the-hilbert-space-into-bond-sectors}}

Similar to the story with the plaquette operators from the spin
language, we can break the Hilbert space \(\mathcal{L}\) up into sectors
labelled by the a set of choices \(\{\pm 1\}\) for the value of each
\(u_{ij}\) operator which I denote by \(\mathcal{L}_u\). Since
\(u_{ij} = -u_{ji}\) we can represent the \(u_{ij}\) graphically with an
arrow that points along each bond in the direction in which
\(u_{ij} = 1\).

Once confined to a particular \(\mathcal{L}_u\), we can `remove the
hats' from the \(\hat{u}_{ij}\) and the hamiltonian becomes a quadratic,
free fermion problem
\[\tilde{H_u} =  \frac{i}{4} \sum_{\langle i,j\rangle_\alpha} 2J^{\alpha} u_{ij} c_i c_j\]
the ground state of which, \(|\psi_u\rangle\) can be found easily via
matrix diagonalisation. If you have been paying very close attention,
you may at this point ask whether the \(\mathcal{L}_u\) are confined
entirely within the physical subspace \(\mathcal{L}_p\) and indeed we
will see that they are not. However it will be helpful to first develop
the theory of the Majorana Hamiltonian a little more.

\begin{figure}
\hypertarget{fig:intro_figure_template}{%
\centering
\includegraphics[width=1\textwidth,height=\textheight]{figure_code/amk_chapter/honeycomb_zoom/intro_figure_template.pdf}
\caption{\textbf{(a)} The standard Kitaev Model is defined on a
honeycomb lattice. The special feature of the honeycomb lattice that
makes the model solveable it is that each vertex is joined by exactly
three bonds i.e the lattice is trivalent. One of three labels is
assigned to each \textbf{(b)} We represent the antisymmetric gauge
degree of freedom \(u_{jk} = \pm 1\) with arrows that point in the
direction \(u_{jk} = +1\) \textbf{(c)} The Majorana transformation can
be visualised as breaking each spin into four Majoranas which then pair
along the bonds. The pairs of x,y and z Majoranas become part of the
classical \(\mathbb{Z}_2\) gauge field \(u_{ij}\) leaving just a single
Majorana \(c_i\) per site.}\label{fig:intro_figure_template}
}
\end{figure}

\hypertarget{the-majorana-hamiltonian}{%
\subsection{The Majorana Hamiltonian}\label{the-majorana-hamiltonian}}

We now have a quadtratic hamiltonian
\[ \tilde{H} =  \frac{i}{4} \sum_{\langle i,j\rangle_\alpha} 2J^{\alpha} u_{ij} c_i c_j\]
in which most of the Majorana degrees of freedom have paired along bonds
to become a classical gauge field \(u_{ij}\). What follows is relatively
standard theory for quadratic Majorana Hamiltonians
\textcite{BlaizotRipka1986}.

As a consequence of the the antisymmetry of the matrix with entries
\(J^{\alpha} u_{ij}\), the eigenvalues of the Hamiltonian
\(\tilde{H}_u\) come in pairs \(\pm \epsilon_m\). This redundant
information is a consequence of the doubling of the Hilbert space which
occured when we transformed to the Majorana representation.

If we pair organise the eigenmodes of \(H\) into pairs such that \(b_m\)
and \(b_m'\) have energies \(\epsilon_m\) and \(-\epsilon_m\) we can
construct the transformation \(Q\)
\[(c_1, c_2... c_{2N}) Q = (b_1, b_1', b_2, b_2' ... b_{N}, b_{N}')\]
and put the Hamiltonian into the form
\[\tilde{H}_u = \frac{i}{2} \sum_m \epsilon_m b_m b_m'\]

The determinant of \(Q\) will be useful later when we consider the
projector from \(\mathcal{\tilde{L}}\) to \(\mathcal{L}\) but otherwise
the \(b_m\) are just an intermediate step. From them we form fermionic
operators \[ f_i = \tfrac{1}{2} (b_m + ib_m')\] with their associated
number operators \(n_i = f^\dagger_i f_i\). These let us write the
Hamiltonian neatly as

\[ \tilde{H}_u = \sum_m \epsilon_m (n_m - \tfrac{1}{2}).\]

The ground state \(|n_m = 0\rangle\) of the many body system at fixed
\(u\) is then \[E_{u,0} = -\frac{1}{2}\sum_m \epsilon_m \] and we can
construct any state from a particular choice of \(n_m = 0,1\).

In cases where all we care about it the value of \(E_{u,0}\) it is
possible to skip forming the fermionic operators. The eigenvalues
obtained directly from diagonalising \(J^{\alpha} u_{ij}\) come in
\(\pm \epsilon_m\) pairs. We can take half the absolute value of the
whole set to recover \(\sum_m \epsilon_m\) easily.

\textbf{The Majorana Hamiltonian is quadratic within a Bond Sector.}

\hypertarget{mapping-back-from-bond-sectors-to-the-physical-subspace}{%
\subsection{Mapping back from Bond Sectors to the Physical
Subspace}\label{mapping-back-from-bond-sectors-to-the-physical-subspace}}

At this point, given a particular bond configuration \(u_{ij} = \pm 1\)
we are able to construct a quadratic Hamiltonian \(\tilde{H}_u\) in the
extended space and diagonalise it to find its ground state
\(|\vec{u}, \vec{n} = 0\rangle\). This is not necessarily the ground
state of the system as a whole, it just the lowest energy state within
the subspace \(\mathcal{L}_u\)

\textbf{However, \(|u, n_m = 0\rangle\) does not lie in the physical
subspace}. As an example let's take the lowest energy state associated
with \(u_{ij} = +1\), this state satisfies
\[u_{ij} |\vec{u}=1, \vec{n} = 0\rangle = |\vec{u}=1, \vec{n} = 0\rangle\]
for all bonds \(i,j\).

If we act on it this state with one of the gauge operators
\(D_j = b_j^x b_j^y b_j^z c_j\) we see that \(D_j\) flips the value of
the three bonds \(u_{ij}\) that surround site \(k\):

\[ |u'\rangle = D_j |u=1, n_m = 0\rangle\]

\[ \begin{aligned}
\langle u'|u_{ij}|u'\rangle &=  \langle u| b_j^x b_j^y b_j^z c_j \;ib^x_i b^x_j\; b_j^x b_j^y b_j^z c_j|u\rangle\\
&= -1
\end{aligned}\]

Since \(D_j\) commutes with the hamiltonian in the extended space
\(\tilde{H}\), the fact that \(D_j\) flips the value of bond operators
is telling us that there is a gauge degeneracy between the ground state
of \(\tilde{H}_u\) and the set of \(\tilde{H}_{u'}\) related to it by
gauge transformations \(D_j\). I.e we can flip any three bonds around a
vertex and the physics will stay the same.

We can turn this into a symmetrisation procedure by taking a
superposition of every possible gauge transformation. Every possible
gauge transformation is just every possible subset of
\({D_0, D_1 ... D_n}\) which can be neatly expressed as
\[|\phi_w\rangle = \prod_i \left( \frac{1 + D_i}{2}\right) |\tilde{\phi}_u\rangle\]
this is nice because the quantity \(\frac{1 + D_i}{2}\) is also the
local projector onto the physical subspace. Here \(|\phi_w\rangle\) is a
gauge invariant state that lives in \(\mathcal{L}_p\) which has been
constructed from a set of states in different \(\mathcal{L}_u\).

This gauge degeneracy leads nicely onto the next topic which is how to
construct a set of gauge invariant quantities out of the \(u_{ij}\),
these will turn out to just be the plaquette operators.

\textbf{The Bond Sectors overlap with the physical subspace but are not
contained within it.}

\hypertarget{properties-of-the-gauge-field}{%
\subsection{Properties of the Gauge
Field}\label{properties-of-the-gauge-field}}

The bond operators \(u_{ij}\) are useful because they label a bond
sector \(\mathcal{\tilde{L}}_u\) in which we can easiy solve the
Hamiltonian. However the gauge operators move us between bond sectors.
\textbf{Bond sectors are not gauge invariant!}

Let's consider instead the properties of the plaquette operators
\(\hat{\phi}_i\) that live on the faces of the lattice.

We already showed that they are conserved. And as one might hope and
expect, the plaquette operators map cleanly on to the bond operators of
the Majorana representation:

\[\begin{aligned}
\tilde{W}_p &= \prod_{\mathrm{i,j}\; \in\; p} \tilde{K}_{ij}\\
            &= \prod_{\mathrm{i,j}\; \in\; p} \tilde{\sigma}_i^\alpha \tilde{\sigma}_j^\alpha\\
            &= \prod_{\mathrm{i,j}\; \in\; p} (ib^\alpha_i c_i)(ib^\alpha_j c_j)\\
            &= \prod_{\mathrm{i,j}\; \in\; p} i u_{ij} c_i c_j\\
            &= \prod_{\mathrm{i,j}\; \in\; p} i u_{ij}
\end{aligned}\]

Where the last steps holds because each \(c_i\) appears exactly twice
and adjacent to its neighbour in each plaquette operator. Note that this
is consistent with the observation from earlier that each \(W_p\) takes
values \(\pm 1\) for even paths and \(\pm i\) for odd paths.

\hypertarget{vortices-and-their-movements}{%
\subsubsection{Vortices and their
movements}\label{vortices-and-their-movements}}

Let's imagine we started from the ground state of the model and flipped
the sign of a single bond. In doing so we will flip the sign of the two
plaquettes adjacent to that bond. I'll call these disturbed plaquettes
\emph{vortices}. I'll refer to a particular choice values for the
plaquette operators as a vortex sector.

If we chain multiple bond flips we can create a pair of vortices at
arbitrary locations. The chain of bonds that we must flip corresponds to
a path on the dual of the lattice.

Something else we can do is create a pair of vortices, move one around a
loop and then anhilate it with its partner. This corresponds to a closed
loop on the dual lattice and applying such a bond flip leaves the vortex
sector unchanged.

Notice that the \(D_j\) operators flip three bonds around a vertex. This
is the smallest closed loop around which one can move a vortex pair and
anhilate it with itself.

Such operations compose in the sense that we can build any larger loop
by applying a series of \(D_j\) operations. Indeed the symetrisation
procedure \(\prod_i \left( \frac{1 + D_i}{2}\right)\) that maps from the
bond sector to a physical state is applying constructing a superposition
over every such loop that leaves the vortex sector unchanged.

The only loops that we cannot build out of \(D_j\)s are non-contractible
loops, such as those that span the major or minor circumference of the
torus.

\textbf{The plaquette operators are the gauge invariant quantity that
determines the physics of the model}

\hypertarget{composition-of-u_jk-loops}{%
\subsubsection{\texorpdfstring{Composition of \(u_{jk}\)
loops}{Composition of u\_\{jk\} loops}}\label{composition-of-u_jk-loops}}

\begin{figure}
\hypertarget{fig:plaquette_addition_by_hand}{%
\centering
\includegraphics[width=0.57\textwidth,height=\textheight]{figure_code/amk_chapter/plaquette_addition/plaquette_addition_by_hand.pdf}
\caption{In the product of individual plaquette operators shared bonds
cancel out. The product is equal to the enclosing
path.}\label{fig:plaquette_addition_by_hand}
}
\end{figure}

Second it is now easy to show that the loops and plaquettes satisfy nice
composition rules, so long as we stick to loops that wind in a
particular direction.

Consider the product of two non-overlapping loops \(W_a\) and \(W_b\)
that share an edge \(u_{12}\). Since the two loops both wind clockwise
and do not overlap, one will contain a term \(i u_{12}\) and the other
\(i u_{21}\). Since the \(u_{ij}\) commute with one another, they square
to \(1\) and \(u_{ij} = -u_{ji}\) we see have \(i u_{12} i u_{21} = 1\)
and we can repeat this for any number of shared edges. Hence, we get a
version of Stokes' theorem: the product of \(i u_{jk}\) around any
closed loop \(\partial A\) is equal to the product of plaquette
operators \(\Phi\) that span the area \(A\) enclosed by that loop:
\[\prod_{u_{jk} \in \partial A} i \; u_{jk} = \prod_{\phi_i \in A} \phi_i\]

\begin{figure}
\hypertarget{fig:stokes_theorem}{%
\centering
\includegraphics[width=0.71\textwidth,height=\textheight]{figure_code/amk_chapter/stokes_theorem/stokes_theorem.pdf}
\caption{The loop composition rule extends to arbitrary numbers of
vortices giving a discrete version of Stoke's
theorem.}\label{fig:stokes_theorem}
}
\end{figure}

\textbf{Wilson loops can always be decomposed into products of
plaquettes operators unless they are non-contractable}

\hypertarget{gauge-degeneracy-and-the-euler-equation}{%
\subsubsection{Gauge Degeneracy and the Euler
Equation}\label{gauge-degeneracy-and-the-euler-equation}}

We can check this analysis with a counting argument. For a lattice with
\(B\) bonds, \(P\) plaquettes and \(V\) vertices we can count how many
bond sectors, vortices sectors and gauge symmetries there are and check
them against Euler's polyhedra equation.

Euler's equation states for a closed surface of genus \(g\), i.e that
has \(g\) holes so \(0\) for the sphere, \(1\) for the torus and \(g\)
for \(g\) tori stuck together \[B = P + V + 2 - 2g\]

\begin{figure}
\hypertarget{fig:torus}{%
\centering
\includegraphics[width=0.86\textwidth,height=\textheight]{figs/torus.jpeg}
\caption{In periodic boundary conditions the Kitaev model is defined on
the surface of a torus. Topologically the torus is distinct from the
sphere in that it has a hole that cannot be smoothly deformed away.
Associated with each such hole are two non-contractible loops on the
surface, here labeled A and B, that cannot be smoothly deformed to a
point. These two non-contracible loops can. be used to construct two
symmetry operators \(\hat{A}\) and \(\hat{A}\) that flip \(u_{jk}\)s
along their paths.}\label{fig:torus}
}
\end{figure}

For the case of the torus where \(g = 1\) we can rearrange this to read:
\[B = (P-1) + (V-1) + 2\]

Each \(u_{ij}\) takes two values and there is one associated with each
bond so there are exactly \(2^B\) distinct configurations of the bond
sector. Let's see if we can factor those configurations out into the
cartesian product of vortex sectors, gauge symmetries and
non-contractible loop operators.

Vortex sectors: each plaquette operator \(\phi_i\) takes two values
(\(\pm 1\) or \(\pm i\)) and there are \(P\) of them so naively one
would think there are \(2^P\). However vortices can only be created on
pairs so there are really \(\tfrac{2^P}{2} = 2^{P-1}\) vortex sectors.

Gauge symmetries: As discussed earlier these correspond to the all
possible compositions of the \(D_j\) operators. Again there are only
\(2^{V-1}\) of these because, as we will see in the next section,
\(\prod_{j} D_j = \mathbb{1}\) in the physical space, and we enforce
this by chooising the correct product of single particle fermion states.
You can get an intuitive picture for why \(\prod_{j} D_j = \mathbb{1}\)
by imagining larger and larger patches of \(D_j\) operators on the
torus. These patches correspond to transporting a vortex pair around the
edge of the patch. At some point the patch wraps around and starts to
cover the entire torus, as this happens the bounday of the patch
disappears and hence it which corresponds to the identity operation. See
Fig ?? (animated in the HTML version).

Finally the torus has two non-contractible loop operators asscociated
with its major and minor diameters.

Putting this all together we see that there are \textbf{\(2^B\) bond
sectors} a space which can be decomposed into the cartesian product of
\textbf{\(2^{P-1}\) vortex sectors}, \textbf{\(2^{V-1}\) gauge
symmetries} and \textbf{\(2^2 = 4\) topological sectors} associated with
the non-contractible loop operators. This last factor forms the basis of
proposals to construct topologically protected qubits since the 4
sectors cold only be mixed by a highly non-local perturbation, ref
?????.

\includegraphics[width=1\textwidth,height=\textheight]{figure_code/amk_chapter/intro/types_of_dual_loops/types_of_dual_loops.pdf}
\includegraphics[width=1\textwidth,height=\textheight]{figure_code/amk_chapter/intro/gauge_symmetries/gauge_symmetries.pdf}
\includegraphics[width=1\textwidth,height=\textheight]{figure_code/amk_chapter/flood_fill/flood_fill.pdf}
\includegraphics[width=1\textwidth,height=\textheight]{figure_code/amk_chapter/flood_fill_amorphous/flood_fill_amorphous.pdf}

\hypertarget{counting-edges-plaquettes-and-vertices}{%
\subsubsection{Counting edges, plaquettes and
vertices}\label{counting-edges-plaquettes-and-vertices}}

It will be useful to know how the trivalent structre of the lattice
constraints the number of bonds \(B\), plaquettes \(P\) and vertices
\(V\) it has.

We can immediately see that the lattice is built from vertices that each
share 3 edges with their neighbours. This means each vertex comes with
\(\tfrac{3}{2}\) bonds i.e \(3V = 2B\). This is consistent with the fact
that in the Majorana representation on the torus each vertex brings
three \(b^\alpha\) operators which then pair along bonds to give \(3/2\)
bonds per vertex.

If we define an integer \(N\) such that \(V = 2N\) and \(B = 3N\) and
substitite this into the polyhedra equation for the torus we see that
\(P = N\). So if is a trivalent lattice on the torus has \(N\)
plaquettes, it has \(2N\) vertices and \(3N\) bonds.

We can also consider the sum of the number of bonds in each plaquette
\(S_p\), since each bond is a member of exactly two plaquettes
\[S_p = 2B = 6N\]

The mean size of a plaquette in a trivalent lattice on the torus is
exactly 6. Since the sum is even, this also tells us that all odd
plaquettes must come in pairs.

\begin{figure}
\hypertarget{fig:hilbert_spaces}{%
\centering
\includegraphics[width=1\textwidth,height=\textheight]{figure_code/amk_chapter/hilbert_spaces.pdf}
\caption{The relationship between the different Hilbert spaces used in
the solution is slightly complex.}\label{fig:hilbert_spaces}
}
\end{figure}

\hypertarget{the-projector}{%
\subsection{The Projector}\label{the-projector}}

It will turn out that the projection from the extended space to the
physical space is not actually that important for the results that I
will present. However it it useful to go through the theory of it to
explain why this is.

The physicil states are defined as those for which
\(D_i |\phi\rangle = |\phi\rangle\) for all \(D_i\). Since \(D_i\) has
eigenvalues \(\pm1\), the quantity \(\tfrac{(1+D_i)}{2}\) has eigenvalue
\(1\) for physical states and \(0\) for extended states so is the local
projector onto the physical subspace.

The global projector is therefore
\[ \mathcal{P} = \prod_{i=1}^{2N} \left( \frac{1 + D_i}{2}\right)\]

for a toroidal trivalent lattice with \(N\) plaquettes \(2N\) vertices
and \(3N\) edges. As I pointed out before the product over \((1 + D_j)\)
can also be thought of as the sum of all possible subsets \(\{i\}\) of
the \(D_j\) operators, which is the set of all possible gauge symmetry
operations.

\[ \mathcal{P} = \frac{1}{2^{2N}} \sum_{\{i\}} \prod_{i\in\{i\}} D_i\]

Since the gauge operators \(D_j\) commute and square to one, we can
define the complement operator \(C = \prod_{i=1}^{2N} D_i\) and see that
it take each set of \(\prod_{i \in \{i\}} D_j\) operators and gives us
the complement of that set. I said earlier that \(C\) is the identity in
the physical subspace and we will shortly see why.

W use the complement operator to rewrite the projector as a sum over
half the subsets \(\{\}\) let's call that \(\Lambda\). The complement
operator deals with the other half

\[ \mathcal{P} =  \left( \frac{1}{2^{2N-1}} \sum_{\Lambda} \prod_{i\in\{i\}} D_i\right) \left(\frac{1 + \prod_i^{2N} D_i}{2}\right) = \mathcal{S} \cdot \mathcal{P}_0\]

To compute \(\mathcal{P}_0\) the main quantity needed is the product of
the local projectors \(D_i\)
\[\prod_i^{2N} D_i = \prod_i^{2N} b^x_i b^y_i b^z_i c_i \] for a
toroidal trivalent lattice with \(N\) plaquettes \(2N\) vertices and
\(3N\) edges.

First we reorder the operators by bond type, this doesn't require any
information about the underlying lattice.

\[\prod_i^{2N} D_i = \prod_i^{2N} b^x_i \prod_i^{2N} b^y_i \prod_i^{2N} b^z_i \prod_i^{2N} c_i\]

The product over \(c_i\) operators reduces to a determinant of the Q
matrix and the fermion parity, see
\textcite{pedrocchiPhysicalSolutionsKitaev2011b} . The only difference
from the honeycomb case is that we cannot explicitely compute the
factors \(p_x,p_y,p_z = \pm\;1\) that arise from reordering the b
operators such that pairs of vertices linked by the corresponding bonds
are adjacent.

\[\prod_i^{2N} b^\alpha_i = p_\alpha \prod_{(i,j)}b^\alpha_i b^\alpha_j\]

However they are simply the parity of the permutation from one ordering
to the other and can be computed in linear time with a cycle
decomposition \textcite{app:cycle_decomp}.

We find that
\[\mathcal{P}_0 = 1 + p_x\;p_y\;p_z\; \mathrm{det}(Q^u) \; \hat{\pi} \; \prod_{\{i,j\}} -iu_{ij}\]

where \(p_x\;p_y\;p_z = \pm 1\) are lattice structure factors. \(Q^u\)
is the determinant of the matrix mentioned earlier that maps \(c_i\)
operators to normal mode operators \(b'_i, b''_i\). These depend only on
the lattice structure. \(\prod -i \; u_{ij}\) depend on the lattice and
the particular vortex sector.
\(\hat{\pi} = \prod{i}^{N} (1 - 2\hat{n}_i)\) is the parity of the
particular many body state determined by fermionic occupation numbers
\(n_i\).

All these factors take values \(\pm 1\) so \(\mathcal{P}_0\) is 0 or 1
for a particular state. Since \(\mathcal{S}\) corresponds to
symmetrising over all the gauge configurations and cannot be 0, this
tells use that once we have determined the single particle eigenstates
of a bond sector, the true many body ground state has the same energy as
either the empty state with \(n_i = 0\) or a state with a single fermion
in the lowest level.

Let's think about where are with the model now. We can map the spin
Hamiltonian to a Majorana Hamiltonian in an extended Hilbert space.
Along with that mapping comes a gauge field \(u_{jk}\) defining
\textbf{bond sectors}. The gauge symmetries of \(u_{jk}\) are generated
by the set of \(D_j\) operators. The gauge invariant and therefore
physically relevant variables are the plaquette operators \(\phi_i\)
which define as a \textbf{vortex sector}. In order to practically solve
the Majorana Hamiltonian we must remove hats from the gauge field by
restricting ourselves to a particular bond sector. From there the
Majorana Hamiltonian becomes non-interacting and we can solve it like
any quadratic theory. This lets us construct the single particle
eigenstates from which we can also construct many body states. However
the many body states constructed this way are not in the physical
subspace!

However for the many body states within a particular bond sector,
\(\mathcal{P}_0 = 0,1\) tells us which of those have some overlap with
the physical sector.

We see that finding a state that has overlap with a physical state only
ever requires the addition or removal of one fermion. There are cases
where this can make a difference but for most observables such as ground
state energy this correction scales away as the number of fermions in
the system grows.

If we wanted to construct a full many body wavefunction in the spin
basis we would need to include the full symmetrisation over the gauge
fields. However this was not necessary for any of the results that will
be presented here.

\hypertarget{open-boundary-conditions}{%
\subsection{Open boundary conditions}\label{open-boundary-conditions}}

Care must be taken in the definition of open boundary conditions. Simply
removing bonds from the lattice leaves behind unpaired \(b^\alpha\)
operators that need to be paired in some way to arrive at fermionic
modes. In order to fix a pairing we always start from a lattice defined
on the torus and generate a lattice with open boundary conditions by
defining the bond coupling \(J^{\alpha}_{ij} = 0\) for sites joined by
bonds \((i,j)\) that we want to remove. This creates fermionic zero
modes \(u_{ij}\) associated with these cut bonds which we set to 1 when
calculating the projector.

Alternatively, since all the fermionic zero modes are degenerate anyway,
an arbitrary pairing of the unpaired \(b^\alpha\) operators could be
performed. \textbf{Is is possible that a lattice constructed and
coloured like this would have unequal numbers of \(b^x\) \(b^y\) and
\(b^z\) operators?}

\hypertarget{the-ground-state-vortex-sector}{%
\subsection{The Ground State Vortex
Sector}\label{the-ground-state-vortex-sector}}

On the Honeycomb, Lieb's theorem implies that the the ground state
corresponds to the state where all \(u_jk = 1\) implying that the flux
free sector is the ground state sector \textcite{lieb_flux_1994}.

Lieb's theorem does not generalise easily to the amorphous case. However
we can get some intuition by examining the problem that will lead to a
guess for the ground state. We will then provide numerical evidence that
this guess is in fact correct.

Let's consider the partition function of the Majorana hamiltonian:
\[ \mathcal{Z} = \mathrm{Tr}\left( e^{-\beta H}\right) = \sum_i \exp{-\beta \epsilon_i}\]
At low temperatures \(\mathcal{Z} \approx \beta \epsilon_0\) where
\(\epsilon_0\) is the lowest energy fermionic state.

How does the \(\mathcal{Z}\) depend on the Majorana hamiltonian?
Expanding the exponential out gives:
\[ \mathcal{Z} = \sum_n \frac{(-\beta)^n}{n!} \mathrm{Tr(H^k)} \]

Now there's an interesting observation to make here. The Hamiltonian is
essentially a scaled adjacency matrix. An adjacency matrix being a
matrix \(g_{ij}\) such that \(g_{ij} = 1\) if vertices \(i\) and \(j\)
and joined by an edge and 0 otherwise.

Powers of adjacency matrices have the property that the entry
\((g^n)_{ij}\) corresponds to the number of paths of length n on the
graph that begin at site \(i\) and end at site \(j\). These include
somewhat degenerate paths that go back on themselves etc.

The trace of an adjacency matrix
\[\mathrm{Tr}(g^n) = \sum_i (g^n)_{ii}\] therefore counts the number
number of loops of size \(n\) that can be drawn on the graph.

Applying the same treatment to our Majorana Hamiltonian, we can
interpret \(u_ij\) to equal 0 if the two sites are not joined by a bond
and we put ourselves in the isotropic phase where \(J^\alpha = 1\)
\[ \tilde{H}_{ij} =  \tfrac{1}{2} i u_{ij}\]

We then see that the trace of the nth power of H is a sums over Wilson
loops of size \(n\) with an additional factor of \(2^{-n}\). We showed
earlier that the Wilson loop operators can always be written as products
of the plaquette operators that they enclose.

Lumping all the prefactors together, we can write:
\[ \mathcal{Z} = c_A \hat{A} + c_B \hat{B} + \sum_i c_i \hat{\phi}_i + \sum_{ij} c_{ij}  \hat{\phi}_i \hat{\phi}_j + \sum_{ijk} c_{ijk}  \hat{\phi}_i \hat{\phi}_j \hat{\phi}_k + ...\]

Where the \(c\) factors would be something like
\[c_{ijk...} = \sum_n \tfrac{(-\beta)^n}{n!} \tfrac{1}{2^n} K_{ijk...}\]
which is a sum over all loop lengths \(n\) and for each we have a
combinatoral factor \(K_{ijk...}\) that counts how many ways there are
to draw a loop of length \(n\) that only encloses plaquettes \(ijk...\).

We also have the pesky non-contractible loop operators \(\hat{A}\) and
\(\hat{B}\). Again the prefactors for these are very complicated but we
can intuitively see that for larger and larger loops lengths there will
be a combinatorial explosion of possible ways that they appear in these
sums. These are suppressed exponentially with system size but at
practical lattice sizes they cause significant finite size effects. The
main evidence of this is that the 4 loop sectors spanned by the
\(\hat{A}\) and \(\hat{B}\) operators are degenerate in the infinite
system size limit, while that degeneracy is lifted in finite sized
systems.

We don't have much hope of actually evaluating this for an amorphous
lattice. However it lead us to guess that the ground state vortex sector
might be a simple function of the side length of each plaquette.

The ground state of the Amorphous Kitaev Model is found by setting the
flux through each plaquette \(\phi\) to be equal to
\(\phi^{\mathrm{g.s.}}(n_{\mathrm{sides}})\)

\[\begin{aligned}
    \phi^{\mathrm{g.s.}}(n_{\mathrm{sides}}) = -(\pm i)^{n_{\mathrm{sides}}},
\end{aligned}\] where \(n_{\mathrm{sides}}\) is the number of edges that
form each plaquette and the choice of sign gives a twofold chiral ground
state degeneracy.

This conjecture is consistent with Lieb's theorem on regular lattices
\autocite{lieb_flux_1994} and is supported by numerical evidence. As
noted before, any flux that differs from the ground state is an
excitation which I call a vortex.

\hypertarget{finite-size-effects}{%
\subsubsection{Finite size effects}\label{finite-size-effects}}

This guess only works for larger lattices because of the finite size
effects. In order to rigorously test it we would like to directly
enumerate the \(2^N\) vortex sectors for a smaller lattice and check
that the lowest state found is the vortex sector predicted by ???.

To do this we tile an amorphous lattice onto a repeating \(NxN\) grid.
The use of a fourier series then allows us to compute the
diagonalisation with a penalty only linear in the number of tiles used
compared to diagonalising a single lattice. With this technique the
finite size effects related to the non-contractible loop operators are
removed with only a linear penalty in computation time compared to the
exponential penalty paid by simply simply diagonalising larger lattices.

Using this technique we verified that \(\phi_0\) correctly predicts the
ground state for hundreds of thousands of lattices with upto 20
plaquettes. For larger lattices we verified that random perturbations
around the predicted ground state never yield a lower energy state.

\hypertarget{chiral-symmetry}{%
\subsection{Chiral Symmetry}\label{chiral-symmetry}}

In the discussion above we see that the ground state has a twofold
\textbf{chiral} degeneracy that comes about because the global sign of
the odd plaquettes does not matter.

This happens because by adding odd plaquettes we have broken the time
reversal symmetry of the original model
\autocite{Chua2011,yaoExactChiralSpin2007,ChuaPRB2011,Fiete2012,Natori2016,Wu2009,Peri2020,WangHaoranPRB2021}.

Similar to the behaviour of the original Kitaev model in response to a
magnetic field, we get two degenerate ground states of different
handedness. Practicaly speaking, one ground state is related to the
other by inverting the imaginary \(\phi\) fluxes
\autocite{yaoExactChiralSpin2007}.

\hypertarget{topology-chirality-and-edge-modes}{%
\subsection{Topology, chirality and edge
modes}\label{topology-chirality-and-edge-modes}}

Most thermodynamic and quantum phases studied can be characterised by a
local order parameter. That is, a function or operator that only
requires knowledge about some fixed sized patch of the system that does
not scale with system size.

However there are quantum phases that cannot be characterised by such a
local order parameter. These phases are intead said to posess
`topological order'.

One property of topological order that is particularly easy to observe
that the ground state degeneracy depends on the topology of the manifold
that we put the system on to. This is referred to as topological
degeneracy to distinguish it from standard symmetry breaking.

The Kitaev model will be a good example of this, we have already looked
at it defined on a graph that is embedded either into the plane or onto
the torus. The extension to surfaces like the torus but with more than
one handle is relatively easy.

\hypertarget{anyonic-statistics}{%
\subsection{Anyonic Statistics}\label{anyonic-statistics}}

In dimensions greater than two, the quantum state of a system must pick
up a factor of \(-1\) or \(+1\) if two identical particles are swapped.
We call these Fermions and Bosons.

This argument is predicated on the idea that performing two swaps is
equivalent to doing nothing. Doing nothing should not change the quantum
state at all, so doing one swap can at most multiply it by \(\pm 1\).

However there are many hidden parts to this argument. Firstly, this
argument just isn't the whole story, if you want to know why Fermions
have half integer spin, for instance, you have to go to field theory.

There is also a second niggle, why does this argument only work in
dimensions greater than two? What we're really saying when we say that
two swaps do nothing is that the world lines of two particles that have
been swapped twice can be untangled without crossing. Why can't they
cross? Well because if they cross then the particles can interact and
the quantum state could change in an arbitrary way. We're implcitly
using the locality of physics here to argue that if the worldlines stay
well separated then the overall quantum state cannot too much.

In two dimensions we cannot untangle the worldlines of two particles
that have swapped places, they are braided together. See Fig
\ref{fig:braiding} for a diagram.

\begin{figure}
\hypertarget{fig:braiding}{%
\centering
\includegraphics[width=0.71\textwidth,height=\textheight]{figure_code/amk_chapter/braiding.png}
\caption{}\label{fig:braiding}
}
\end{figure}

From this fact flows a whole new world of behaviours, now the quantum
state can aquire a phase factor \(e^{i\phi}\) upon exchange of two
identical particles, which we now call Anyons.

The Kitaev Model is a good demonstration of the connection beween Anyons
and topological degeneracy. In the Kitaev model we can create a pair of
vortices, move one around a non-contractable loop \(\mathcal{T}_{x/y}\)
and then anhilate them together. Without topology this should leave the
quantum state unchanged. Instead it moves us to another ground state in
a topologically degenerate ground state subspace. Practically speaking
it flips a dual line of bonds \(u_{jk}\) going around the loop which we
cannot undo with any gauge transformation made from \(D_j\) operators.

If the ground state subspace is multidimensional, quasiparticle exchange
can move us around in the space with an action corresponding to a
matrix. These matrices do not in general commmute and so these are known
as non-Abelian anyons.

From here things get even more complex, the Kitaev model has a
non-Abelian phase when exposed to a magnetic field, and the amorphous
Kitaev Model has a non-Abelian phase because of its broken chiral
symmetry.

The way that we have subdivided the Kitaev model into vortex sectors, we
have a neat separation beween vortices and fermionic excitations.
However if we looked at the full many body picture we would see that a
vortex caries with it a cloud of bound majorana states.

\begin{figure}
\hypertarget{fig:majorana_bound_states}{%
\centering
\includegraphics[width=1\textwidth,height=\textheight]{figure_code/amk_chapter/majorana_bound_states/majorana_bound_states.pdf}
\caption{(Left) A large amorphous lattice in the ground state save for a
single pair of vortices shown in red, separated by the string of bonds
that we flipped to create them. (Right) The density of the lowest energy
Majorana state in this vortex sector. The state is clearly bound to the
vortices.}\label{fig:majorana_bound_states}
}
\end{figure}

Consider two processes

\begin{enumerate}
\def\labelenumi{\arabic{enumi})}
\item
  We transport one half of a vortex pair around either the x or y loops
  of the torus before anhilating back to the ground state vortex sector
  \(\mathcal{T}_{x,y}\).
\item
  We flip a line of bond operators coresponding to measuring the flux
  through either the major or minor axes of the torus
  \(\mathcal{\Phi}_{x,y}\)
\end{enumerate}

\begin{figure}
\hypertarget{fig:loops_and_dual_loops}{%
\centering
\includegraphics[width=1.14\textwidth,height=\textheight]{figure_code/amk_chapter/loops_and_dual_loops/loops_and_dual_loops.pdf}
\caption{(Left) The two topological flux operators of the toroidal
lattice, these don't correspond to any face of the lattice, but rather
measure flux that threads through the major and minor axes of the torus.
This shows a particular choice but any loop that crosses the boundary is
gauge equivalent to one of or the sum of these two loop. (Right) The two
ways to transport vortices around the diameters. These correspond to
creating a vortex pair, transporting one of them around the major or
minor diameters of the torus and then anhilating them
again.}\label{fig:loops_and_dual_loops}
}
\end{figure}

The plaquette operators \(\phi_i\) are associated with fluxes. Wilson
loops that wind the torus are associated with the fluxes through its two
diameters \(\mathcal{\Phi}_{x,y}\).

In the Abelian phase we can move a vortex along any path we like and
then when we bring them back together they will anhilate back to the
vacuum, where we understand `the vacuum' to refer to one of the ground
states, though not necesarily the same one we started in. We can use
this to get from the \((\Phi_x, \Phi_y) = (+1, +1)\) ground state and
construct the set \((+1, +1), (+1, -1), (-1, +1), (-1, -1)\).

\begin{figure}
\hypertarget{fig:topological_fluxes}{%
\centering
\includegraphics[width=0.57\textwidth,height=\textheight]{figure_code/amk_chapter/topological_fluxes.png}
\caption{Wilson loops that wind the major or minor diameters of the
torus measure flux winding through the hole of the donut/torus or
through the filling. If they made donuts that had both a jam filling and
a hole this analogy would be a lot easier to make
\textcite{parkerWhyDoesThis}.}\label{fig:topological_fluxes}
}
\end{figure}

However in the non-Abelian phase we have to wrangle with monodromy
\autocite{chungExplicitMonodromyMoore2007,oshikawaTopologicalDegeneracyNonAbelian2007}.
Monodromy is behaviour of objects as they move around a singularity.
This manifests here in that the identity of a vortex and cloud of
Majoranas can change as we wind them around the torus in such a way that
rather than anhilating to the vacuum the anhilate to create an excited
state instead of a ground state. This means we end up with only three
degenerate ground states in the non-Abelian phase
\((+1, +1), (+1, -1), (-1, +1)\)
\autocite[yaoAlgebraicSpinLiquid2009a]{chungTopologicalQuantumPhase2010}.
The way that this shows up concretly is that the projector enforces both
flux and fermion parity. When we wind a vortex around both
non-contractible loops of the torus, it flips the flux parity which
forces means we have to introduce a fermionic excitation to make the
state physical. Hence the process does not give a fourth ground state.

\includegraphics[width=0.86\textwidth,height=\textheight]{figure_code/amk_chapter/threefold_degeneracy.png}
\includegraphics[width=1.14\textwidth,height=\textheight]{figure_code/amk_chapter/state_decomposition_animated/state_decomposition_animated.pdf}

One reason the topology has gained interest recently is there have
proposals to use this ground state degeneracy to implement both
passively fault tolerant and actively stabilised quantum computations
{[}\textcite{kitaevFaulttolerantQuantumComputation2003};
\textcite{poulinStabilizerFormalismOperator2005};
hastingsDynamicallyGeneratedLogical2021{]}.
