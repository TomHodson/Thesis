\begin{Shaded}
\begin{Highlighting}[]
\OperatorTok{\%\%}\NormalTok{html}
\end{Highlighting}
\end{Shaded}

One of the most interesting and perhaps surprising features of many body
physics is the existence of distinct phases of matter.

Why does liquid turn to ice? Well there are two key ingredients. First
we need a system composed of a large number of objects and second we
need those objects to interact with eachother.

It turns out that the more objects there are, the greater the effect
that their interactions has on the whole. A hundred \(H_2O\) molecules
can't actually form the nice regular structure that characterises ice,
instead you'd get more of a blob. However, any human scale amount of
water contains an unimaginable huge number of molecules.

Phases come about when the interactions between individuals components
serve the reinforce

When a many body, interacting system can display radically different
properties depending on the system parameters

\hypertarget{themes}{%
\subsection{Themes}\label{themes}}

\begin{itemize}
\item
  many body
\item
  interactions
\item
  quantum
\item
  topology
\item
  disorder
\item
  quasiparticles
\item
  topological order
\item
  protected edge states
\item
  abelian and non-abelian anyons
\item
  localisation
\item
  lengthscales
\end{itemize}

\begin{Shaded}
\begin{Highlighting}[]
\NormalTok{\_\_ Connection between }
\end{Highlighting}
\end{Shaded}
