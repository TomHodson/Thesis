\begin{Shaded}
\begin{Highlighting}[]
\OperatorTok{\%\%}\NormalTok{html}
\end{Highlighting}
\end{Shaded}

One of the most interesting and perhaps surprising features of many body physics is the existence of distinct phases of matter.

Why does liquid turn to ice? Well there are two key ingredients. First we need a system composed of a large number of objects and second we need those objects to interact with eachother.

It turns out that the more objects there are, the greater the effect that their interactions has on the whole. A hundred \(H_2O\) molecules can't actually form the nice regular structure that characterises ice, instead you'd get more of a blob. However, any human scale amount of water contains an unimaginable huge number of molecules.

Phases come about when the interactions between individuals components serve the reinforce

When a many body, interacting system can display radically different properties depending on the system parameters

\hypertarget{themes}{%
\subsection{Themes}\label{themes}}

\begin{itemize}
\item
  many body
\item
  interactions
\item
  quantum
\item
  topology
\item
  disorder
\item
  quasiparticles
\item
  topological order
\item
  protected edge states
\item
  abelian and non-abelian anyons
\item
  localisation
\item
  lengthscales
\end{itemize}

\hypertarget{condsened-matter-systems}{%
\subsection{Condsened Matter Systems}\label{condsened-matter-systems}}

\hypertarget{spin-orbit-coupling}{%
\subsubsection{Spin-Orbit Coupling}\label{spin-orbit-coupling}}

Electronic wavefunctions can be understood as quantum extensions of

This can be loosely understood as a consequence of that fact that electrons are `orbiting' their host nucleus and in doing so they are moving with respect to an electric field generated by the positive charge of the nucleus. The electric field looks like a magnetic field in the rest frame of the electron and this magnetic field couples to the magnetic spin moment of the electron.

This analogy is wrong on many levels but it suffices to understand that there should be such an effect.

Going one level deeper we can estimate the scale of the effect by combining the non-relativistic quantum theory of a spin in a magnetic field with the classical relativistic electromagnetism prediction for how the electric field turns into a magnetic field in the rest frame of the electron. This gets us within a factor to two of the correct answer but it fails to account for an extra relativistic effect called Thomas Precession \textbf{cite}.

The next level would be to compute this effect within relativistic QM using the Dirac equation. And finally, we could do the full calculation within Quantum Electrodynamics where we would find tiny corrections that come about from virtual processes involving particle-antiparticle pairs that spring form from the vacuum.

\hypertarget{electronic-correlations-the-hubbard-model}{%
\subsubsection{Electronic correlations: The Hubbard Model}\label{electronic-correlations-the-hubbard-model}}

\begin{fignos:no-prefix-figure-caption}

\begin{figure}
\centering
\includegraphics{5d575ef5-9414-4f30-a2cc-9a2b8cd44cc0.png}
\caption{image.png}
\end{figure}

\end{fignos:no-prefix-figure-caption}

These are easiest to understand within the context of the Hubbard model, if we take spin \(1/2\) fermions hopping on the lattice with hopping parameter \(t\) and interaction strength \(U\) \[ H = -t \sum_{\langle i,j \rangle \alpha} c^\dagger_{i\alpha} c_{j\alpha} + \sum_i c^\dagger_{i\uparrow} c_{i\downarrow}\]

where \(c^\dagger_{i\alpha}\) creates a spin \(\alpha\) electron at site \(i\). Pauli exclusion prevents two electrons with the same spin being at the same site so which is why the interaction term only couples opposite spin electrons. The only physically relevant parameter here is \(U/t\) which compared the interaction strength \(U\) to the importance of kinetic energy \(t\).

In the free fermion limit \(U/t = 0\), we can just find the single particle eigenstates and fill them up to the fermi level. The many body ground state has no particular electron-electron correlations.

In the interacting limit, \(t/U = 0\), there's no hopping so electrons just site wherever we put them. We can fill the system up until there is one electron per site without any energy penalty at all. The maximum we can fill the system up to

\begin{fignos:no-prefix-figure-caption}

\begin{figure}
\centering
\includegraphics{f25fb28d-4239-4184-9a9e-b6704189019d.png}
\caption{Stolen from https://arxiv.org/pdf/1701.07056.pdf}
\end{figure}

\end{fignos:no-prefix-figure-caption}

\begin{Shaded}
\begin{Highlighting}[]

\end{Highlighting}
\end{Shaded}
