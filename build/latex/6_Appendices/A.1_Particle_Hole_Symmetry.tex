\hypertarget{particle-hole-symmetry}{%
\section{Particle-Hole Symmetry}\label{particle-hole-symmetry}}

The Hubbard and FK models on a bipartite lattice have particle-hole (PH) symmetry \(\mathcal{P}^\dagger H \mathcal{P} = - H\), accordingly they have symmetric energy spectra. The associated symmetry operator \(\mathcal{P}\) exchanges creation and annihilation operators along with a sign change between the two sublattices. In the language of the Hubbard model of electrons \(c_{\alpha,i}\) with spin \(\alpha\) at site \(i\) the particle hole operator corresponds to the substitution of new fermion operators \(d^\dagger_{\alpha,i}\) and number operators \(m_{\alpha,i}\) where

\[d^\dagger_{\alpha,i} = \epsilon_i c_{\alpha,i}\] \[m_{\alpha,i} = d^\dagger_{\alpha,i}d_{\alpha,i}\]

the lattices must be bipartite because to make this work we set \(\epsilon_i = +1\) for the A sublattice and \(-1\) for the even sublattice~\autocite{gruberFalicovKimballModel2005}.

The entirely filled state \(\ket{\Omega} = \sum_{\alpha,i} c^\dagger_{\alpha,i} \ket{0}\) becomes the new vacuum state \[d_{i\sigma} \ket{\Omega} = (-1)^i c^\dagger_{i\sigma} \sum_{j\rho} c^\dagger_{j\rho} \ket{0} = 0.\]

The number operator \(m_{\alpha,i} = 0,1\) counts holes rather than electrons \[ m_{\alpha,i} = c_{\alpha,i} c^\dagger_{\alpha,i} = 1 - c^\dagger_{\alpha,i} c_{\alpha,i}.\]

With the last equality following from the fermionic commutation relations. In the case of nearest neighbour hopping on a bipartite lattice this transformation also leaves the hopping term unchanged because \(\epsilon_i \epsilon_j = -1\) when \(i\) and \(j\) are on different sublattices: \[ d^\dagger_{\alpha,i} d_{\alpha,j} = \epsilon_i \epsilon_j c_{\alpha,i} c^\dagger_{\alpha,j} = c^\dagger_{\alpha,i} c_{\alpha,j} \]

Defining the particle density \(\rho\) as the number of fermions per site: \[
    \rho = \frac{1}{N} \sum_i \left( n_{i \uparrow} + n_{i \downarrow} \right)
\]

The PH symmetry maps the Hamiltonian to itself with the sign of the chemical potential reversed and the density inverted about half filling: \[ \text{PH} : H(t, U, \mu) \rightarrow H(t, U, -\mu) \] \[ \rho \rightarrow 2 - \rho \]

The Hamiltonian is symmetric under PH at \(\mu = 0\) and so must all the observables, hence half filling \(\rho = 1\) occurs here. This symmetry and known observable acts as a useful test for the numerical calculations.
