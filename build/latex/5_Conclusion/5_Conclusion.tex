The FK model

FK model as a way to probe the Mott insulator state. Also the Mott insulator gives rise to the QSl and the doped Mott Insulator may be the source of the sought after High-\(T_c\) superconductor. The concept of quantum orders is relevant because for instance, if we can classify the kinds of order in the MI state, we can classify the kinds of high-\(T_c\) theories that can emerge from them.

As we discussed in the introduction, Quantum spin liquids can arise out of Mott insulators and both are highly interaction dependent. However interactions are hard to study. Using exactly solvable systems like the FK and KH model as a way to look at the physics of many-body interacting systems. Exactly solveable models are a useful part of the theoretical toolbox for investigating these systems.

\textbf{Role of dimensionality}

\textbf{Role of Correlations} Another theme from the two models is that longer range correlations from criticality in the LRFK model and anti-correlations in the topological disorder in the AK model, lead to a wider range of effects that short range correlations.

\textbf{Role of disorder}

paragraph about topological order as new addition to the pantheon of spontaneously broken symmetries

Xiao-Gang Wen~\autocite{wenQuantumOrdersSymmetric2002} when talks about quantum orders as a those that arise within quantum states at zero temperature, included QSLs, FQH states and superconductors\footnote{Wen argues that superconductors cannot be characterised be a local order parameter in the way that superfluids can.}. He also argues that the High-\(T_c\) superconductors is in terms of them being doped Mott insulators so that we should try to understand the QSL which emerges from the undoped Mott insulator (at half filling).

The existence of distinct, spatially limited quasiparticle excitations is not obvious.

emergent gauge physics, could condensed matter systems be useful in understanding the standard model too?

Electron-electron interactions play a dominant role in determining electronic and thermodynamic properties in these strongly correlated materials.

Specific examples where strongly correlated materials may figure prominently are high temperature superconductors and hard magnets without rare earth elements.

\hypertarget{material-realisations}{%
\section{Material Realisations}\label{material-realisations}}

\hypertarget{amorphous-materials}{%
\subsection{Amorphous Materials}\label{amorphous-materials}}

\hypertarget{metal-organic-frameworks}{%
\subsection{Metal Organic Frameworks}\label{metal-organic-frameworks}}

\hypertarget{discussion}{%
\section{Discussion}\label{discussion}}
