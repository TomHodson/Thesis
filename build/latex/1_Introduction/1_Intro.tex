\hypertarget{interacting-quantum-many-body-systems}{%
\section{Interacting Quantum Many Body Systems}\label{interacting-quantum-many-body-systems}}

When you take many objects and let them interact together, it is often simpler to describe the behaviour of the group differently from the way one would describe the individual objects. Consider a flock of starlings like that of \cref{fig:Studland_Starlings}. Watching the flock you'll see that it has a distinct outline, that waves of density will sometimes propagate through the closely packed birds and that the flock seems to respond to predators as a distinct object. The natural description of this phenomena is couched in terms of the flock rather than of the individual birds.

The behaviours of the flock are an \emph{emergent phenomena}. The starlings are only interacting with their immediate six or seven neighbours~\autocite{king2012murmurations,balleriniInteractionRulingAnimal2008}, what a physicist would call a \emph{local interaction}. There is much philosophical debate about how exactly to define emergence~\autocite{andersonMoreDifferent1972,kivelsonDefiningEmergencePhysics2016} but for our purposes it enough to say that emergence is the fact that the aggregate behaviour of many interacting objects may necessitate a description very different from that of the individual objects.

\hypertarget{fig:Studland_Starlings}{%
\begin{figure}
\centering
\includegraphics[width=1\textwidth,height=\textheight]{figure_code/intro_chapter/Studland_Starlings.jpeg}
\caption[{A murmuration of Starlings}]{A murmuration of starlings. Dorset, UK. Credit \href{https://twitter.com/arripay}{Tanya Hart}, ``Studland Starlings'', 2017, \href{https://creativecommons.org/licenses/by-sa/3.0/deed.en}{CC BY-SA 3.0}}
\label{fig:Studland_Starlings}
\end{figure}
}

To give an example closer to the topic at hand, our understanding of thermodynamics began with bulk properties like heat, energy, pressure and temperature~\autocite{saslowHistoryThermodynamicsMissing2020}. It was only later that we gained an understanding of how these properties emerge from microscopic interactions between very large numbers of particles~\autocite{flammHistoryOutlookStatistical1998}.

Condensed Matter is, at its heart, the study of what behaviours emerge from large numbers of interacting quantum objects at low energy. When these three properties are present together: a large number of objects, those objects being quantum and there are interaction between the objects, we call it an interacting quantum many body system. From these three ingredients nature builds all manner of weird and wonderful materials.

Historically, we made initial headway in the study of many-body systems, ignoring interactions and quantum properties. The ideal gas law and the Drude classical electron gas~\autocite{ashcroftSolidStatePhysics1976} are good examples. Including interactions into many-body physics leads to the Ising model~\autocite{isingBeitragZurTheorie1925}, Landau theory~\autocite{landau2013fluid} and the classical theory of phase transitions~\autocite{jaegerEhrenfestClassificationPhase1998}. In contrast, condensed matter theory got it state in quantum many-body theory. Bloch's theorem~\autocite{blochÜberQuantenmechanikElektronen1929} predicted the properties of non-interacting electrons in crystal lattices, leading to band theory. In the same vein, advances were made in understanding the quantum origins of magnetism, including ferromagnetism and antiferromagnetism~\autocite{MagnetismCondensedMatter}.

The development of Landau-Fermi Liquid theory explained why band theory works so well even in cases where an analysis of the relevant energies suggests that it should not~\autocite{wenQuantumFieldTheory2007}. Landau Fermi Liquid theory demonstrates that in many cases where electron-electron interactions are significant, the system can still be described in terms on generalised non-interacting quasiparticles.

However there are systems where even Landau Fermi Liquid theory fails. An effective theoretical description of these systems must include electron-electron correlations and they are thus called Strongly Correlated Materials~\autocite{morosanStronglyCorrelatedMaterials2012}, Correlated Electron systems or Quantum Materials. The canonical examples are superconductivity~\autocite{MicroscopicTheorySuperconductivity}, the fractional quantum hall effect~\autocite{feldmanFractionalChargeFractional2021} and the Mott insulators~\autocite{mottBasisElectronTheory1949,fisherMottInsulatorsSpin1999}. We'll start by looking at the latter but shall see that there are many links between three topics.

\hypertarget{mott-insulators}{%
\section{Mott Insulators}\label{mott-insulators}}

Mott Insulators are remarkable because their electrical insulator properties come from electron-electron interactions. Electrical conductivity, the bulk movement of electrons, requires both that there are electronic states very close in energy to the ground state and that those states are delocalised so that they can contribute to macroscopic transport. Band insulators are systems whose Fermi level falls within a gap in the density of states and thus fail the first criteria. Band insulators derive their character from the characteristics of the underlying lattice. Anderson Insulators have only localised electronic states near the fermi level and therefore fail the second criteria. We will discuss Anderson insulators and disorder in a later section.

Both band and Anderson insulators occur without electron-electron interactions. Mott insulators, by contrast, require a many body picture to understand and thus elude band theory and single-particle methods.

\hypertarget{fig:venn_diagram}{%
\begin{figure}
\centering
\includegraphics[width=1\textwidth,height=\textheight]{figure_code/intro_chapter/venn_diagram}
\caption[{Interacting Quantum Many Body Systems Venn Diagram}]{Three key adjectives. Many Body, the fact of describing systems in the limit of large numbers of particles. Quantum, objects whose behaviour requires quantum mechanics to describe accurately. Interacting, the constituent particles of the system affect one another via forces, either directly or indirectly. When taken together, these three properties can give rise to what are called strongly correlated materials.}
\label{fig:venn_diagram}
\end{figure}
}

The theory of Mott insulators developed out of the observation that many transition metal oxides are erroneously predicted by band theory to be conductive~\autocite{boerSemiconductorsPartiallyCompletely1937} leading to the suggestion that electron-electron interactions were the cause of this effect~\autocite{mottDiscussionPaperBoer1937}. Interest grew with the discovery of high temperature superconductivity in the cuprates in 1986~\autocite{bednorzPossibleHighTcSuperconductivity1986} which is believed to arise as the result of a doped Mott insulator state~\autocite{leeDopingMottInsulator2006}.

The canonical toy model of the Mott insulator is the Hubbard model~\autocite{gutzwillerEffectCorrelationFerromagnetism1963,kanamoriElectronCorrelationFerromagnetism1963,hubbardj.ElectronCorrelationsNarrow1963} of \(1/2\) fermions hopping on the lattice with hopping parameter \(t\) and electron-electron repulsion \(U\)

\[ H_{\mathrm{H}} = -t \sum_{\langle i,j \rangle \alpha} c^\dagger_{i\alpha} c_{j\alpha} + U \sum_i n_{i\uparrow} n_{i\downarrow} - \mu \sum_{i,\alpha} n_{i\alpha}\]

where \(c^\dagger_{i\alpha}\) creates a spin \(\alpha\) electron at site \(i\) and the number operator \(n_{i\alpha}\) measures the number of electrons with spin \(\alpha\) at site \(i\). The sum runs over lattice neighbours \(\langle i,j \rangle\) including both \(\langle i,j \rangle\) and \(\langle j,i \rangle\) so that the model is Hermition.

In the non-interacting limit \(U << t\), the model reduces to free fermions and the many-body ground state is a separable product of Bloch waves filled up to the Fermi level. In the interacting limit \(U >> t\) on the other hand, the system breaks up into a product of local moments, each in one the four states \(|0\rangle, |\uparrow\rangle, |\downarrow\rangle, |\uparrow\downarrow\rangle\) depending on the filing.

The Mott insulating phase occurs at half filling \(\mu = \tfrac{U}{2}\) where there is one electron per lattice site~\autocite{hubbardElectronCorrelationsNarrow1964}. Here the model can be rewritten in a symmetric form \[ H_{\mathrm{H}} = -t \sum_{\langle i,j \rangle \alpha} c^\dagger_{i\alpha} c_{j\alpha} + U \sum_i (n_{i\uparrow} - \tfrac{1}{2})(n_{i\downarrow} - \tfrac{1}{2})\]

The basic reason that the half filled state is insulating seems is trivial. Any excitation must include states of double occupancy that cost energy \(U\), hence the system has a finite bandgap and is an interaction driven Mott insulator. Depending on the lattice, the local moments may then order antiferromagnetically. Originally it was proposed that this antiferromagnetic order was the cause of the gap opening~\autocite{mottMetalInsulatorTransitions1990}. However, Mott insulators have been found~\autocite{law1TTaS2QuantumSpin2017,ribakGaplessExcitationsGround2017} without magnetic order. Instead the local moments may form a highly entangled state known as a quantum spin liquid, which will be discussed shortly.

Various theoretical treatments of the Hubbard model have been made, including those based on Fermi liquid theory, mean field treatments, the local density approximation (LDA)~\autocite{slaterMagneticEffectsHartreeFock1951} and dynamical mean-field theory~\autocite{greinerQuantumPhaseTransition2002}. None of these approaches are perfect. Strong correlations are poorly described by the Fermi liquid theory and the LDA approaches while mean field approximations do poorly in low dimensional systems. This theoretical difficulty has made the Hubbard model a target for cold atom simulations~\autocite{mazurenkoColdatomFermiHubbard2017}.

From here the discussion will branch two directions. First, we will discuss a limit of the Hubbard model called the Falikov-Kimball Model. Second, we will look at quantum spin liquids and the Kitaev honeycomb model.

\textbf{The Falikov-Kimball Model}

Though not the original reason for its introduction, the Falikov-Kimball (FK) model is the limit of the Hubbard model as the mass ratio of the spin up and spin down electron is taken to infinity. This gives a model with two fermion species, one itinerant and one entirely immobile. The number operators for the immobile fermions are therefore conserved quantities and can be be treated like classical degrees of freedom. For our purposes it will be useful to replace the immobile fermions with a classical Ising background field \(S_i = \pm1\).

\[\begin{aligned}
H_{\mathrm{FK}} = & -\;t \sum_{\langle i,j \rangle} c^\dagger_{i}c_{j} + \;U \sum_{i} S_i\;(c^\dagger_{i}c_{i} - \tfrac{1}{2}). \\ 
\end{aligned}\]

Given that the physics of states near the metal-insulator (MI) transition is still poorly understood~\autocite{belitzAndersonMottTransition1994,baskoMetalInsulatorTransition2006} the FK model provides a rich test bed to explore interaction driven MI transition physics. Despite its simplicity, the model has a rich phase diagram in \(D \geq 2\) dimensions. It shows an Mott insulator transition even at high temperature, similar to the corresponding Hubbard Model~\autocite{brandtThermodynamicsCorrelationFunctions1989}. In 1D, the ground state phenomenology as a function of filling can be rich~\autocite{gruberGroundStatesSpinless1990} but the system is disordered for all \(T > 0\)~\autocite{kennedyItinerantElectronModel1986}. The model has also been a test-bed for many-body methods, interest took off when an exact dynamical mean-field theory solution in the infinite dimensional case was found~\autocite{antipovCriticalExponentsStrongly2014,ribicNonlocalCorrelationsSpectral2016,freericksExactDynamicalMeanfield2003,herrmannNonequilibriumDynamicalCluster2016}.

In Chapter 3 I will introduce a generalized FK model in one dimension. With the addition of long-range interactions in the background field, the model shows a similarly rich phase diagram. I use an exact Markov chain Monte Carlo method to map the phase diagram and compute the energy-resolved localization properties of the fermions. I then compare the behaviour of this transitionally invariant model to an Anderson model of uncorrelated binary disorder about a background charge density wave field which confirms that the fermionic sector only fully localizes for very large system sizes.

\hypertarget{quantum-spin-liquids}{%
\section{Quantum Spin Liquids}\label{quantum-spin-liquids}}

To turn to the other key topic of this thesis, we have discussed the question of the magnetic ordering of local moments in the Mott insulating state. The local moments may form an AFM ground state. Alternatively they may fail to order even at zero temperature~\autocite{law1TTaS2QuantumSpin2017,ribakGaplessExcitationsGround2017}, giving rise to what is known as a quantum spin liquid (QSL) state.

Landau-Ginzburg-Wilson theory characterises phases of matter as inextricably linked to the emergence of long range order via a spontaneously broken symmetry. The fractional quantum Hall (FQH) state, discovered in the 1980s is an explicit example of an electronic system that falls outside of the Landau-Ginzburg-Wilson paradigm. FQH systems exhibit fractionalised excitations linked to their ground state having long range entanglement and non-trivial topological properties~\autocite{broholmQuantumSpinLiquids2020}. Quantum spin liquids are the analogous phase of matter for spin systems. Remarkably the existence of QSLs was first suggested by Anderson in 1973~\autocite{andersonResonatingValenceBonds1973}.

\hypertarget{fig:correlation_spin_orbit_PT}{%
\begin{figure}
\centering
\includegraphics[width=1\textwidth,height=\textheight]{figure_code/intro_chapter/correlation_spin_orbit_PT.png}
\caption[{Phase Diagram}]{From~\autocite{TrebstPhysRep2022}.}
\label{fig:correlation_spin_orbit_PT}
\end{figure}
}

The main route to QSLs, though there are others~\autocite{balentsNodalLiquidTheory1998,balentsDualOrderParameter1999,linExactSymmetryWeaklyinteracting1998}, is via frustration of spin models that would otherwise order have AFM order. This frustration can come geometrically, triangular lattices for instance cannot support AFM order. It can also come about as a result of spin-orbit coupling.

Electron spin naturally couples to magnetic fields. Spin-orbit coupling is a relativistic effect, that very roughly corresponds to the fact that in the frame of reference of a moving electron, the electric field of nearby nuclei look like magnetic field to which the electron spin couples. In certain transition metal based compounds, such as those based on Iridium and Rutheniun, crystal field effects, strong spin-orbit coupling and narrow bandwidths lead to effective spin-\(\tfrac{1}{2}\) Mott insulating states with strongly anisotropic spin-spin couplings known as Kitaev Materials~\autocite{Jackeli2009,HerrmannsAnRev2018,Winter2017,TrebstPhysRep2022,Takagi2019}. Kitaev materials draw their name from the celebrated Kitaev Honeycomb Model as it is believed they will realise the QSL state via the mechanisms of the Kitaev Model.

The Kitaev Honeycomb model~\autocite{kitaevAnyonsExactlySolved2006} was the first concrete model with a QSL ground state. It is defined on the honeycomb lattice and provides an exactly solvable model whose ground state is a QSL characterized by a static \(\mathbb Z_2\) gauge field and Majorana fermion excitations. It can be reduced to a free fermion problem via a mapping to Majorana fermions which yields an extensive number of static \(\mathbb Z_2\) fluxes tied to an emergent gauge field. The model is remarkable not only for its QSL ground state, it supports a rich phase diagram hosting gapless, Abelian and non-Abelian phases~\autocite{knolleDynamicsFractionalizationQuantum2015} and a finite temperature phase transition to a thermal metal state~\autocite{selfThermallyInducedMetallic2019}. It been proposed that its non-Abelian excitations could be used to support robust topological quantum computing~{[}\textcite{kitaev_fault-tolerant_2003}; \textcite{freedmanTopologicalQuantumComputation2003}; nayakNonAbelianAnyonsTopological2008{]}.

It is by now understood that the Kitaev model on any tri-coordinated \(z=3\) graph has conserved plaquette operators and local symmetries~\autocite{Baskaran2007,Baskaran2008} which allow a mapping onto effective free Majorana fermion problems in a background of static \(\mathbb Z_2\) fluxes~\autocite{Nussinov2009,OBrienPRB2016,yaoExactChiralSpin2007,hermanns2015weyl}. However, depending on lattice symmetries, finding the ground state flux sector and understanding the QSL properties can still be challenging~\autocite{eschmann2019thermodynamics,Peri2020}.

\textbf{paragraph about amorphous lattices}

In Chapter 4 I will introduce a soluble chiral amorphous quantum spin liquid by extending the Kitaev honeycomb model to random lattices with fixed coordination number three. The model retains its exact solubility but the presence of plaquettes with an odd number of sides leads to a spontaneous breaking of time reversal symmetry. I unearth a rich phase diagram displaying Abelian as well as a non-Abelian quantum spin liquid phases with a remarkably simple ground state flux pattern. Furthermore, I show that the system undergoes a finite-temperature phase transition to a conducting thermal metal state and discuss possible experimental realisations.

\hypertarget{outline}{%
\section{Outline}\label{outline}}

The next chapter, Chapter 2, will introduce some necessary background to the Falikov-Kimball Model, the Kitaev Honeycomb Model, disorder and localisation.

In Chapter 3 I introduce the Long Range Falikov-Kimball Model in greater detail. I will present results that. Chapter 4 focusses on the Amorphous Kitaev Model.
