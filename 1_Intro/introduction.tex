A short one page opener before thesis outline.


\section{Thesis Outline}
This thesis is composed of two main studies of separate but related physical models, The Falikov-Kimball Model and the Kitaev-Honeycomb Model. In this chapter I will discuss the overarching motivations for looking at these two physical models. I will then review the literature and methods that are common to both models.

In Chapter 2 I will look at the Falikov-Kimball model. I will review what it is and why we would want to study it. I'll survey what is already known about it and identify the gap in the research that we aim to fill, namely the model's behaviour in one dimension. I'll then introduce the modified model that we came up with to close this gap. I will present our results on the thermodynamic phase diagram and localisation properties of the model

In Chapter 3 I'll study the Kitaev Honeycomb Model, following the same structure as Chapter 2 I will motivate the study, survey the literature and identify a gap. I'll introduce our Amorphous Kitaev Model designed to fill this gap and present the results.

Finally in chapter 4 I will summarise the results and discuss what implications they have for our understanding interacting many-body quantum systems.

\section{Introduction}
\subsection{Condensed Matter Theory}

\fig{energy_scales}{A collection of important energy scales plotted on a logarithmic scale. We can take room temperature somewhere in the middle of the plot as a reference point. Moving towards higher energies we'll gradually probe deeper into the smaller and smaller structures of the universe, atoms, protons, quarks etc. While going to lower energies we move to larger and larger scales and objects made up of progressively more and more components.}

There's a common joke among condensed matter physicists that the field is the study of dirt. I quite like this definition because within the average handful of dirt you're not unlikely to find crystals, metals, glasses and all manner of other things that condensed matter physicists like to study. 

The second reason I like this definition is that dirt is complex. It has structure at many scales. It is heterogeneous and contaminated with traces of all manner of things. It's dirty. When I talk about real physical things like this, I will use the word systems.

On the other hand, we have mathematical tools at our disposal to describe and understand the world around us. I'll use the word model whenever I am talking about these mathematical constructions that describe the real word. 

This leads one of the central tensions of condensed matter theory: our models are ill-equipped to deal with the complexity inherent in the real world. We may be able to write down a mathematical description of a handful of dirt, but we wouldn't be able to solve it. And a model we cannot solve will not further our understanding.

So what do we do? Well, we simplify. We try to ask "What are the essential ingredients required to model this particular phenomenon?"

Answering this question lets us write down simpler models that we have a better chance of being able to solve. And by paring the description down to the bare minimum, it also gives us a better chance of being able to understand why these systems behave the way they do.

I will therefore be focusing on highly simplified models of reality, each containing only a few ingredients. I will then investigate what phenomena can be observed in these models and try to relate the findings to more messy reality.

\subsection{Interacting Many-Body Quantum Systems}

[Figure showing energy scales of interest on a log plot]

Condensed matter is called condensed because in the scheme of things it is the physics of objects that are quite cold. Working at high energies allows one to probe deeper and deeper into the substructure of things, from atoms to nuclei to nucleons to quarks to gluons. At one stage the behaviour may be well described by talking of protons and neutrons binding together, while at the next it is necessary to bring gluons and quarks into the picture.

It is perhaps not surprising then, that going to lower energies produces the reverse effect. If we start with a gas of atoms and cool it, we will get a solid. In that solid, we find that the relevant objects are not even particles in the sense that one might imagine, they are instead collective motions (or excitations) of the constituent particles which we call quasiparticles.

Sometimes it is said that quasiparticles are a bit like waves on the ocean. And in a sense they are, however because they live in the quantum realm they are far weirder and more diverse than ocean waves. Their quantumness gives them a distinct identity and indivisibility that waves do not have. They can interact both directly and, more strangely, simply by moving around one another. 

I will be looking at the behaviour of electrons in crystalline and amorphous materials. The ingredients that I will consider are: 

\begin{markdown}
- Interactions between particles
- Disorder or randomness in the system
- Symmetries of the system
- The dimensionality of the system
- Temperature
- While perhaps obvious, the fact of there being many bodies is itself a key ingredient.
\end{markdown}

In keeping with the theme of trying to keep things simple, I will use the tight binding approximation of electrons moving against a fixed atomic background potential. I will discuss the advantages and disadvantages of this approximation in the next chapter but for now let's say that it captures enough of the essence of reality to explain many physical phenomena.

\subsection{Models with classical and quantum degrees of freedom}

\subsection{Tight Binding Hamiltonians and their Symmetries}

\subsection{Phase Transitions}

\subsection{Disorder and Localisation}

\subsection{The Falikov-Kimball Model}

\subsection{The Kitaev Honeycomb Model}

\subsection{Research Aims}
%  \fig{venn_diagram}{Modelling reality requires three basic ingredients: we need quantum objects, we need a lot of them and we need them to interact. However these three properties taken together make for an almost completely intractable problem. Many approaches to modelling reality in condensed matter can be thought of as tackling this problem by going after one of these ingredients. Diagrammatic methods set interactions to zero and add them back in perturbatively. Mean field methods get rid of the many bodies. Condensed matter theorists are excited about ADS/CFT correspondences because the can map interacting quantum problems onto classical ones. The models I study in this thesis could be classified into this scheme as follows. The FK model model contains many interacting bodies but what makes it tractable is that we have classical species interacting with quantum ones, because the quantum particles never interact among one another we can still make progress with simulations. The Kitaev Honeycomb Model and our generalisations of it essentially rely on a similar trick, though in this case the classical degrees of freedom are emergent they still serve to decouple the quantum degrees of freedom from interacting with one another.}



