%stuff I've added
\usepackage[utf8]{inputenc}
\usepackage{graphicx}% Include figure files
\DeclareGraphicsExtensions{.pdf,.png,.jpg,.jpeg}

% font setup
% \usepackage{libertine}
% \usepackage[libertine]{newtxmath}

% for control of the headers and footers
\usepackage{fancyhdr} 
\pagestyle{fancy}
\fancyhf{}
\fancyhead[L]{\ifthenelse{\isodd{\value{page}}}{\leftmark}{\rightmark}}
\fancyfoot[C]{\thepage}
\setlength{\headheight}{15pt}
\addtolength{\topmargin}{-2.5pt}

\usepackage{tikz}
% \usepackage{dcolumn}% Align table columns on decimal point
% \usepackage{bm}% bold math
\usepackage{acronym}
\usepackage{physics}
% \usepackage{layouts}
\usepackage{acronym}
\usepackage{lipsum} 

\usepackage{amsmath}
\usepackage{amssymb}
\usepackage{svg}
\usepackage{comment}
\usepackage{placeins}

\usepackage{float}
\floatplacement{figure}{t} % the default figure placement specifier

\usepackage{hyperref}
\usepackage{cleveref}
\usepackage{bookmark}

\usepackage{pdfpages} % To be able to include pdfs in the appendix

\usepackage[pages=some]{background}


\usepackage[
backend=biber, %use the biber backed which is supposed to be better than the bibtex backend
style=phys, %use the AIP style defined by the biblatex-phys package
citestyle=numeric-comp, %for the inline citations use compressed numeric citations ie [1], [1,3,5] [4-8]
biblabel=brackets, %the style used in bibliography itself ie [78-80]
url=true, eprint=true, doi=true, %biblatex-phys options, whether to include these or not
sorting=none, %sort in order of citation
hyperref=true, %try to make urls into links and emit a warning if the hyperref package is not explicitely loaded
maxnames=4, %after this many names truncate
maxnames=4, %when truncating, how any names to include
backref=true, % add back references to the bibliography, showing where each reference is used
backrefstyle=two, %in the back references, compress any two or more consecutive pages to a range 1-5
]{biblatex}
\addbibresource{entire_zotero_library.bib}

\def\[#1\]{\begin{equation}#1\end{equation}}

% Extra symbol commands
% \newcommand{\tex}[1]{\expval{#1}} %Thermal expectation value
% \newcommand{\qex}[1]{\expval{#1}} %Quantum expectation value
% \newcommand{\Z}{\mathcal{Z}} %Partition function
% \newcommand{\T}{\mathcal{T}} %MCMC Transition function
% \newcommand{\A}{\mathcal{A}} %MCMC Acceptance function
% \newcommand{\p}{\mathcal{P}} %MCMC Proposal distribution
% \newcommand{\nfi}{n^f_i} %FK number operators of f electrons
% \newcommand{\nfj}{n^f_j} %FK number operators of f electrons
% \newcommand{\s}{\vec{s}} %used to refer to states of the spin system

\usepackage{orcidlink}

%for code blocks
\usepackage{listings}
\usepackage{xcolor}

\definecolor{codegreen}{rgb}{0,0.6,0}
\definecolor{codegray}{rgb}{0.5,0.5,0.5}
\definecolor{codepurple}{rgb}{0.58,0,0.82}
\definecolor{backcolour}{rgb}{0.95,0.95,0.92}
\definecolor{urlblue}{HTML}{007bff}

\lstdefinestyle{mystyle}{
    backgroundcolor=\color{backcolour},   
    commentstyle=\color{codegreen},
    keywordstyle=\color{magenta},
    numberstyle=\tiny\color{codegray},
    stringstyle=\color{codepurple},
    basicstyle=\ttfamily\footnotesize,
    breakatwhitespace=false,         
    breaklines=true,                 
    captionpos=b,                    
    keepspaces=true,                 
    numbers=left,                    
    numbersep=5pt,                  
    showspaces=false,                
    showstringspaces=false,
    showtabs=false,                  
    tabsize=2
}

\lstset{style=mystyle}
%endfor code blocks

\hypersetup{
    colorlinks = true,
    linkcolor  = black,
    citecolor  = black,
    urlcolor   = urlblue,
}

\urlstyle{same}

\usepackage{wrapfig}
\usepackage{floatflt}

%stuff that came in the template

\usepackage{graphicx}
\usepackage{verbatim}
\usepackage{latexsym}
\usepackage{mathchars}
\usepackage{setspace}

\pagestyle{empty}

%

\makeatletter  %to avoid error messages generated by "\@". Makes Latex treat "@" like a letter

% \linespread{1.5} % enable to go back to 1.5 line spacing
\def\submitdate#1{\gdef\@submitdate{#1}}

\def\maketitle{
  \begin{titlepage}{
    %\linespread{1.5}
    \Large Imperial College of Science, Technology and Medicine \\
    %\linebreak
    Department of Physics
    \rm
    \vskip 3in
    \Large \bf \@title \par
  }
  \vskip 0.3in
  \par
  {\Large \@author}
  \vskip 4in
  \par
  Submitted in part fulfilment of the requirements for the degree of 
  \linebreak
  Doctor of Philosophy in Physics of Imperial College of Science, Technology and Medicine \@submitdate
  \vfil
  \end{titlepage}
}

\def\titlepage{
  \newpage
  \centering
  \linespread{1}
  \normalsize
  \vbox to \vsize\bgroup\vbox to 9in\bgroup
}
\def\endtitlepage{
  \par
  \kern 0pt
  \egroup
  \vss
  \egroup
  \cleardoublepage
}

\def\abstract{
  \pagenumbering{arabic}
  \begin{center}{
    \large\bf Abstract}
  \end{center}
  \small
  %\def\baselinestretch{1.5}
  % \linespread{1.5}
  \normalsize
}
\def\endabstract{
  \par
}

\newenvironment{acknowledgements}{
  \cleardoublepage
  \begin{center}{
    \large \bf Acknowledgements}
  \end{center}
  \small
  % \linespread{1.5}
  \normalsize
}{\cleardoublepage}
\def\endacknowledgements{
  \par
}

\newenvironment{dedication}{
  \cleardoublepage
  \begin{center}{
    \large \bf Dedication}
  \end{center}
  \small
  % \linespread{1.5}
  \normalsize
}{\cleardoublepage}
\def\enddedication{
  \par
}

\def\preface{
    % \pagenumbering{roman}
    \pagestyle{plain}
    % \doublespacing
}

\def\body{
    \pagestyle{plain}
    \cleardoublepage    
    \setlength{\parskip}{1.1ex}
    \tableofcontents
    % \cleardoublepage
    % \pagestyle{uheadings}
    % \listoftables
    % \pagestyle{plain}
    % \cleardoublepage

    \listoffigures
    \cleardoublepage

    \pagestyle{fancy}
    \setlength{\parskip}{2ex plus 0.5ex minus 0.2ex}
    \setlength{\parindent}{0pt}
}

\makeatother  %to avoid error messages generated by "\@". Makes Latex treat "@" like a letter

\newcommand{\ipc}{{\sf ipc}}

\newcommand{\Prob}{\bbbp}
\newcommand{\Real}{\bbbr}
% \newcommand{\real}{\Real}
\newcommand{\Int}{\bbbz}
\newcommand{\Nat}{\bbbn}


% Properly styled differentiation and integration operators
\newcommand{\diff}[1]{\mathrm{\frac{d}{d\mathit{#1}}}}
\newcommand{\diffII}[1]{\mathrm{\frac{d^2}{d\mathit{#1}^2}}}
\newcommand{\intg}[4]{\int_{#3}^{#4} #1 \, \mathrm{d}#2}
\newcommand{\intgd}[4]{\int\!\!\!\!\int_{#4} #1 \, \mathrm{d}#2 \, \mathrm{d}#3}

% Large () brackets on different lines of an eqnarray environment
\newcommand{\Leftbrace}[1]{\left(\raisebox{0mm}[#1][#1]{}\right.}
\newcommand{\Rightbrace}[1]{\left.\raisebox{0mm}[#1][#1]{}\right)}

% Funky symbols for footnotes
\newcommand{\symbolfootnote}{\renewcommand{\thefootnote}{\fnsymbol{footnote}}}
% now add \symbolfootnote to the beginning of the document...

\newcommand{\normallinespacing}{\renewcommand{\baselinestretch}{1.5} \normalsize}
\newcommand{\mediumlinespacing}{\renewcommand{\baselinestretch}{1.2} \normalsize}
\newcommand{\narrowlinespacing}{\renewcommand{\baselinestretch}{1.0} \normalsize}
\newcommand{\bump}{\noalign{\vspace*{\doublerulesep}}}
\newcommand{\cell}{\multicolumn{1}{}{}}
\newcommand{\spann}{\mbox{span}}
\newcommand{\diagg}{\mbox{diag}}
\newcommand{\modd}{\mbox{mod}}
\newcommand{\minn}{\mbox{min}}
\newcommand{\andd}{\mbox{and}}
\newcommand{\forr}{\mbox{for}}
\newcommand{\EE}{\mbox{E}}

\newcommand{\deff}{\stackrel{\mathrm{def}}{=}}
\newcommand{\syncc}{~\stackrel{\textstyle \rhd\kern-0.57em\lhd}{\scriptstyle L}~}

\def\coop{\mbox{\large $\rhd\!\!\!\lhd$}}
\newcommand{\sync}[1]{\raisebox{-1.0ex}{$\;\stackrel{\coop}{\scriptscriptstyle
#1}\,$}}

\newtheorem{definition}{Definition}[chapter]
\newtheorem{theorem}{Theorem}[chapter]

%%% For things that pandoc uses
\newenvironment{Shaded}{}{}
\newenvironment{Highlighting}{}{}
\newcommand{\AlertTok}[1]{\textcolor[rgb]{1.00,0.00,0.00}{\textbf{#1}}}
\newcommand{\AnnotationTok}[1]{\textcolor[rgb]{0.38,0.63,0.69}{\textbf{\textit{#1}}}}
\newcommand{\AttributeTok}[1]{\textcolor[rgb]{0.49,0.56,0.16}{#1}}
\newcommand{\BaseNTok}[1]{\textcolor[rgb]{0.25,0.63,0.44}{#1}}
\newcommand{\BuiltInTok}[1]{#1}
\newcommand{\CharTok}[1]{\textcolor[rgb]{0.25,0.44,0.63}{#1}}
\newcommand{\CommentTok}[1]{\textcolor[rgb]{0.38,0.63,0.69}{\textit{#1}}}
\newcommand{\CommentVarTok}[1]{\textcolor[rgb]{0.38,0.63,0.69}{\textbf{\textit{#1}}}}
\newcommand{\ConstantTok}[1]{\textcolor[rgb]{0.53,0.00,0.00}{#1}}
\newcommand{\ControlFlowTok}[1]{\textcolor[rgb]{0.00,0.44,0.13}{\textbf{#1}}}
\newcommand{\DataTypeTok}[1]{\textcolor[rgb]{0.56,0.13,0.00}{#1}}
\newcommand{\DecValTok}[1]{\textcolor[rgb]{0.25,0.63,0.44}{#1}}
\newcommand{\DocumentationTok}[1]{\textcolor[rgb]{0.73,0.13,0.13}{\textit{#1}}}
\newcommand{\ErrorTok}[1]{\textcolor[rgb]{1.00,0.00,0.00}{\textbf{#1}}}
\newcommand{\ExtensionTok}[1]{#1}
\newcommand{\FloatTok}[1]{\textcolor[rgb]{0.25,0.63,0.44}{#1}}
\newcommand{\FunctionTok}[1]{\textcolor[rgb]{0.02,0.16,0.49}{#1}}
\newcommand{\ImportTok}[1]{#1}
\newcommand{\InformationTok}[1]{\textcolor[rgb]{0.38,0.63,0.69}{\textbf{\textit{#1}}}}
\newcommand{\KeywordTok}[1]{\textcolor[rgb]{0.00,0.44,0.13}{\textbf{#1}}}
\newcommand{\NormalTok}[1]{#1}
\newcommand{\OperatorTok}[1]{\textcolor[rgb]{0.40,0.40,0.40}{#1}}
\newcommand{\OtherTok}[1]{\textcolor[rgb]{0.00,0.44,0.13}{#1}}
\newcommand{\PreprocessorTok}[1]{\textcolor[rgb]{0.74,0.48,0.00}{#1}}
\newcommand{\RegionMarkerTok}[1]{#1}
\newcommand{\SpecialCharTok}[1]{\textcolor[rgb]{0.25,0.44,0.63}{#1}}
\newcommand{\SpecialStringTok}[1]{\textcolor[rgb]{0.73,0.40,0.53}{#1}}
\newcommand{\StringTok}[1]{\textcolor[rgb]{0.25,0.44,0.63}{#1}}
\newcommand{\VariableTok}[1]{\textcolor[rgb]{0.10,0.09,0.49}{#1}}
\newcommand{\VerbatimStringTok}[1]{\textcolor[rgb]{0.25,0.44,0.63}{#1}}
\newcommand{\WarningTok}[1]{\textcolor[rgb]{0.38,0.63,0.69}{\textbf{\textit{#1}}}}
\providecommand{\tightlist}{%
  \setlength{\itemsep}{0pt}\setlength{\parskip}{0pt}}

% For Pandoc
% Scale images if necessary, so that they will not overflow the page
% margins by default, and it is still possible to overwrite the defaults
% using explicit options in \includegraphics[width, height, ...]{}
\setkeys{Gin}{width=\maxwidth,height=\maxheight,keepaspectratio}